\classheader{2018-08-15}
How to solve the n-th order linear ODE?
\subsection*{Homogeneous Equations with Constant Coefficients}
\begin{align*}
	\underbrace{a_n(t) y^{(n)} + a_{n-1}(t) y^{(n-1)} + \cdots + a_1(t) y^1 + a_0(t)y}_{L[y]} & = G(t)\\
	\overbrace{y^{(n)} + p_1(t) y^{(n-1)} + \cdots + p_{n-1}(t)y' + p_n(t)y} = & g(t)
\end{align*}
\textbf{Answer:} Same as before is the short answer:\\
The homogenous part $(L[y] = 0)$, if thhe coefficients are constants, can be solved by exponentials: Assume $L[e^{\Gamma t}] = 0$ to construct the characteristics equation:
\begin{equation*}
	a_n \Gamma^n + a_{n-1} \Gamma^{n-1} + \cdots + a_1\Gamma + a_0 = 0 \tag{$\star$}
\end{equation*}
The roots of $(\star)$ correspond to solutions $y(t) = e^{\Gamma t}$ which are solutions to $L[y] = 0$.\\\\
The rest of the theory also:
\begin{enumerate}[label=\protect\circled{\arabic*}]
	\item If roots of $(\star)$ can be found all of them, counting multiplicity and complex conjugates, one can construct an $n$-parameter family of solutions the fundamental set of solutions:
	\begin{example-N}
		Suppose $(\star)$ has all red distinct roots $\Gamma_1 \neq \Gamma_1 \neq \cdots \neq \Gamma_n$: Then $y(t) = c_1e^{\Gamma_1 t} + \cdots + c_ne^{\Gamma_n t}$ is the general solution.
	\end{example-N}
	\begin{example-N}
		For repeated roots, the pattern is similar to the 2nd order version.
		\begin{enumerate}[label=\protect\circled{\alph*}]
		\item Suppose characteristics equation of a 5th order ODE where $(\Gamma - 2)(\Gamma + 1)^3(\Gamma - 5) = 0$ then $\Gamma_1 = 2$, $\Gamma_2 = \Gamma_3 = \Gamma_4 = -1$, $\Gamma_5 = 5$, and 
		\begin{equation*}
			y(t) = c_1e^{2t} + c_2 e^{-t} + c_3te^{-t} + c_4t^2e^{-t} + c_5e^{5t}
		\end{equation*}
		\item Suppose $(\Gamma^2 - 6)(\Gamma^2 - 4\Gamma + 13)^2 = 0$
		\begin{gather*}
			\Rightarrow \Gamma_1 = \sqrt{6}\\ \Gamma_2 = -\sqrt{6}\\ \Gamma_3 = \Gamma_5 = 2+3i\\
			\Gamma_4 = \Gamma_6 = 2-3i\\
			\text{and } y(t) = c_1e^{\sqrt{6}t} + c_2e^{-\sqrt{6}t} + e^{2t}(c_3 \cos 3t + c_4 \sin 3t) + te^{2t}(c_5 \cos 3t + c_6 \sin 3t)
		\end{gather*}
		\end{enumerate}	
	\end{example-N}
	Solution methods for non-homogenous linear nth order ODEs are the same:
	\begin{enumerate}[label=\protect\circled{\roman*}]
		\item Undetermined Coefficients - exactly the same as the 2nd order version
		\item Variation of Parameters
	\end{enumerate}
	\item Variation of Parameters \\
	Assume  $\Ylines(t) = u_1y_1 + \cdots + u_ny_n$ for $y_1, \cdots, y_n$ solutions to the homogeneous version.\\
	Playing the same game by taking derivatives, making assumptions (to simplicity) and plugging into the ODE, one obtaining a set of $n$ equations.
	\begin{gather*}
		u_1'y_1 + \cdots + u_n'y_n = 0\\
		u_1'y_1' + \cdots + u_n'y_n' = 0\\
		u_1'y_1'' + \cdots + u_n'y_n'' = 0\\
		\vdots \quad \quad \vdots \quad \quad \vdots\\
		u_1'y_1^{(n-1)} + \cdots + u_n'y_n^{n-1} = g(t)\\
	\end{gather*}
	This set of $n$-equations in $n$-unknowns (the derivatives of the $u_i$'s) can be solved and the solution will be unique if $W(y_1, \cdots, y_n) \neq 0$
\end{enumerate}
And lastly, Reduction of Order methods work perfectly well:\\
Assume $y_1(t)$ solves an nth order linear homogeneous ODE. Then $y_2(0) = v(t)y_1(t)$ leads to a $(n-1)$th order ODE in $v'$. Not necessarily easier, but perhaps $(n-1)$th order ODE may have obvious solutions?