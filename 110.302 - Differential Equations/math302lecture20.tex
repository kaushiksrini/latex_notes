\classheader{2018-08-20}
\subsection*{Matrices Review}
A linear system of equations looks like
\begin{equation*}
	\underline{n eqns.}
	\begin{cases}
		a_{11}x_1 + \cdots + a_{1n}x_n = b_1\\
		a_{21}x_1 + \cdots + a_{2n}x_n = b_2\\
		\vdots \quad \quad \quad \ddots \quad \quad \vdots \quad \quad \quad  \vdots\\
		\underbrace{a_{n1}x_1 + \cdots + a_{nn}x_n = b_n}_{\text{n-unknowns}}\\
	\end{cases}
\end{equation*}
We can write this as a single (matrix) equation by collecting up the constituent parts into arrays.
\begin{equation*}
	\underbrace{\begin{bmatrix}
		a_{11} & \cdots & a_{1n}\\
		a_{21} & \cdots & a_{2n}\\
		\vdots & \ddots & \vdots\\
		a_{n1} & \cdots & a_{nn}\\
	\end{bmatrix}}_{A_{nxn}}
	\underbrace{\begin{bmatrix}
		x_1\\
		x_2\\
		\vdots\\
		x_n		
	\end{bmatrix}}_{\vec{X}_{nx1}}
	=
	\underbrace{\begin{bmatrix}
		b_1\\
		b_2\\
		\vdots\\
		b_n		
	\end{bmatrix}}_{\vec{b}_{nx1}}
\end{equation*}
Here $b_2$ = (row 2 of A) $\cdot  \vec{x}$ where the determinant is matrix multiplication.
\subsubsection*{Some facts about matrices and matrix equations}
\begin{enumerate}[label=\protect\circled{\arabic*}]
	\item If $\vec{b} = \begin{bmatrix}
		0\\ \vdots \\ 0 
	\end{bmatrix}$ in $A\vec{x} = \vec{b}$, the equation is called homogenous.
	\item A solution to $A\vec{x} = \vec{b}$ is a choice of $\vec{x}$ which satisfies the equation.
	\item If $\det A \neq 0$, the system has a unique solution.
	\item If $A\vec{x} = \vec{0}$ and $\det A \neq 0$, then $\vec{x} = \begin{bmatrix}
		0\\ \vdots \\ 0 
	\end{bmatrix}$ is the only solution.\\
	If $\det A = 0$, then tons of solutions ($A\vec{x} = \vec{0}$ is never inconsistent)
	\item If $A\vec{x} \neq \vec{0}$, then the inverse of A, $A^{-1}$ exists and can be used to "solve" $A\vec{x} = \vec{b}$: $A\vec{x} = \vec{b} \Rightarrow \vec{x} = A^{-1}\vec{b}$. There is also an identity matrix $I_n  = \begin{bmatrix} \begin{smallmatrix}
		1 & 0 & \cdots & 0\\
		0 & 1 & \cdots & \vdots\\
		\vdots & \vdots & \ddots & 0\\
		0 & \cdots & 0 & 1
		\end{smallmatrix}
	\end{bmatrix} =$ n-dim Identity matrix
	\item The idea of solving a system of equations involves adding multiples of equations to other equations in order to produce new simpler equations.\\
	In matrices, these are the elementary row operations one performs to A to reduce the number of non-zero entries. But what one does to A, one must also do to $\vec{I}$. We need to use an augmentation matrix.
	\item All vectors, by convention, are considered column vectors. To talk about a row vector
	\begin{equation*}
		\begin{split}
			\vec{x} = \begin{bmatrix}
				x_1\\
				\vdots\\
				x_n
			\end{bmatrix}
		\end{split}
		\quad \quad \quad
		\begin{split}
			\vec{x}^T = \begin{bmatrix}
				x_1, \cdots, x_n
			\end{bmatrix}
		\end{split}
	\end{equation*}
	are should either specify "row vector", or take the transpose of a column vector.
	\begin{definition}
		A set of vectors $\vec{x}^{(1)}, \cdots, \vec{x}^{(n)}$ (careful of the notation) of the same size are said to be \underline{linearly dependent} (on each other) if $\exists$ a real numbers $c_1, \cdots, c_n \in \mathbb{R}$, not all at 0, where
		\begin{equation*}
			c_1\vec{x}^{(1)} + \cdots + c_n\vec{x}^{(n)} = 0
		\end{equation*}
		Otherwise they are \underline{linearly independent}\\
		\textbf{Note: } The columns of $A_{nxn}$ are \underline{linearly independent} iff $\det A \neq 0$
	\end{definition}
	\item For $A\vec{x} = \vec{b}$, think of $\vec{x}, \vec{b} \in \mathbb{R}^n$ where $\mathbb{R}^n = $ the set of all n-vectors.\\ 
	Then an $nxn$ matrix $A_{nxn}$ can be considered a linear transformation of $\mathbb{R}^n$ (a function taking $\mathbb{R}^n$ to $\mathbb{R}^n$) taking $\vec{x}$ to $\vec{b} = A\vec{x}$:
	\begin{gather*}
		A: \mathbb{R}^n \longrightarrow \mathbb{R}^n\\
		\vec{x} \xmapsto{A} \vec{b} = A\vec{x}
	\end{gather*}
	$A$ takes n-vectors to n-vectors, where $\vec{b}$ is the image of $\vec{x}$ under $A$.
	\item There is a special equation in linear algebra:
	\begin{equation*}
		\begin{split}
			\boxed{A\vec{x} = \lambda \vec{x}}
		\end{split}
		\quad \quad \quad
		\begin{split}
			A_{nxn} \text{ - matrix}\\
			\vec{x}\text{ - n-vector}\\
			\lambda \text{ - scalar}
		\end{split}
	\end{equation*}
	A choice of $\vec{x}$ and $\lambda$ which satisfy this equation indicate a direction (of $\vec{x}$) unchanged via multiplication by $A$, and expanded or contracted by a factor $\lambda$.\\
	Here $\vec{x}$ is called an \underline{eigenvector} of $A$, and $\lambda$ is its corresponding \underline{eigenvalue}.\\\\
	How to find eigenvalues?
	\begin{gather*}
	A\vec{x} = \lambda\vec{x}\\
	A\vec{x} - \lambda\vec{x} = \vec{0}\\
	A\vec{x} - \lambda I_n \vec{x} = \vec{0}\\
	(A - \lambda I_n)\vec{x} = \vec{0}\\	
	\end{gather*}
	The \underline{only} way non-trivial solutions exist is if $\det (A-\lambda I_n) = 0$. But this equation \underline{only} has $\lambda$ in it!
	\begin{example-N}
		Let $A = \begin{bmatrix}
			1 & 1\\ 6 & 0
		\end{bmatrix}$. Then\\
		\begin{gather*}
			\det (A - \lambda I_2) = 0 = \det (
			\begin{bmatrix}
				\begin{smallmatrix}
					1 & 1\\ 6 & 0
				\end{smallmatrix}
			\end{bmatrix} - 
			\begin{bmatrix}
				\begin{smallmatrix}
					\lambda & 0\\ 0 & \lambda
				\end{smallmatrix}
			\end{bmatrix}) = 
			\begin{bmatrix}
				\begin{smallmatrix}
					1 - \lambda & 1\\ 6 & -\lambda
				\end{smallmatrix}
			\end{bmatrix}\\
			= (1 - \lambda)(-\lambda) - 6 = 0 = (\lambda - 3)(\lambda + 2)\\
			\boxed{\lambda = 3, \lambda = -2}
		\end{gather*}
	\end{example-N}
\end{enumerate}