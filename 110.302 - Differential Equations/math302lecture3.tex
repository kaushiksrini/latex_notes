\classheader{2018-06-16}
Very generally, a first-order ODE of the form
\begin{equation*}
	\dfrac{dy}{dt} = f(t, y)
\end{equation*}
will have a function of both $t$ and $y$ and will not be solvable. However, with some additional structure to $f$, thhere are methods to solve:  In chapter 2, we explore some of these.

\section*{First Order Differential Equations}
\subsection*{Linear Equations}
Suppose $f(t, y) = -p(t)y+q(t)$ for some function $p(t), q(t)$. Then $(\star)$ can be rewritten
\begin{equation*}
	\tcbhighmath[drop fuzzy shadow]{y' = -p(t)y+q(t) \quad \text{\textbf{or}} \quad
	y'+p(t)y= q(t)} \tag{$\star\star$}
\end{equation*}
This new form exposes a structure that facilitates calculation: The LHS is almost the total derivative of a function. To make of so, we multiply the ODE by an expression called the integrating factor.
\begin{definition-N}
	An \underline{integrating factor} is a term that when multiplied to an expression renders the expression integrable.
\end{definition-N}
\redhline\\
\newcommand{\intfac}{e^{\int p(t) dt}}
To understand what we are looking for, look at the patterns here:\\
Let y be a differentiable function of $t$. Then, for any other diff. function of $t$, $f(t)$, we have
\begin{equation*}
	\dfrac{d}{dt}\bigg[f(t)y\bigg] = f(t)y' + f'(t)y \quad \quad \textit{(by Product Rule)}
\end{equation*}
And also,
\begin{equation*}
	\begin{split}
		\dfrac{d}{dt}\bigg[e^{f(t)}y\bigg] & = \mathlarger{e^{f(t)}y' + e^{f(t)}f'(t)y}\\
		& = \mathlarger{e^{f(t)}\bigg[y' + f'(t)y\bigg]}
	\end{split}
\end{equation*}
We do this just to look for patterns. In this case, we see an important one: Inside the brackets, $[y' + f'(t)y]$ looks very close to the LHS of $y'+p(t)y= q(t)$\\
In fact, they are precisely the same when $\mathlarger{f'(t) = p(t)}$, or $\mathlarger{f(t) = \int p(t)dt}$\\
\redhline\\\\
So we do more calculation for a pattern:
\begin{equation*}
	\begin{split}
		\dfrac{d}{dt}\bigg[\mathlarger{e^{\int p(t)dt}y}\bigg] & = e^{\int p(t)dt}y' + \dfrac{d}{dt}\bigg[e^{\int p(t)dt}\bigg] y\\
		& = e^{\int p(t)dt}y' + e^{\int p(t)dt}p(t)y\\
		& = e^{\int p(t) dt}\bigg[ \underbrace{y' + p(t)y}_{\text{precisely the LHS of $(\star\star)$}}\bigg]
	\end{split}
\end{equation*}
This is useful because, if we take $y' + p(t)y = q(t)$ and multiply it the entire equation by $e^{\int p(t)dt}$, then the LHS becomes \underline{easily integrable}.\\
\begin{center}
	We call $e^{\int p(t)dt}$ the \underline{integrating factor} of $y' + p(t)y = q(t)$.
\end{center}
\redhline\\\\
\textbf{\large Solve $y' + p(t)y = q(t)$}
\begin{enumerate}[label=\textbf{Step \arabic*:}]
	\setlength{\itemindent}{0.8in}
	\item Multiply the entire equation by $e^{\int p(t) dt}$.\\
	\begin{equation*}
		e^{\int p(t) dt}\Big[y' + p(t)y = q(t)\Big]
	\end{equation*}
	\begin{equation*}
		\underbrace{e^{\int p(t) dt}y' + e^{\int p(t) dt} p(t)y} = e^{\int p(t) dt} q(t)
	\end{equation*}
	\begin{equation*}
		\dfrac{d}{dt}\bigg[e^{\int p(t) dt}y\bigg] = e^{\int p(t) dt} q(t)
	\end{equation*}
	\item Integrate with respect to (w.r.t.) $t$.
	\begin{equation*}
		\int \dfrac{d}{dt}\big[e^{\int p(t) dt}y\big]dt = \int e^{\int p(t) dt}q(t) dt
	\end{equation*}
	\begin{equation*}
		e^{\int p(t) dt}y = \int e^{\int p(t) dt} q(t) dt + c
	\end{equation*}
	\item Solve for $y$.
	\begin{equation*}
		y(t) = e^{-\int p(t) dt}\bigg[\int e^{\int p(t) dt} q(t) dt + c\bigg]
	\end{equation*}
\end{enumerate}
\redhline\\\\
\textbf{\large Notes:}
\begin{enumerate}[label=\protect\circled{\arabic*}]
	\item Theoretically, we can \underline{always} do this. Practically, the integrating factor $e^{\int p(t) dt}$ is pretty easy to calculate usually.
	\item You do not need to memorise anything of the form of step 3. Just remember the steps.
	\item Any antiderivative of p(t) will do since \circled{a} they all only differ by a constant and \circled{b} You are multiplying the entire equation by the factor. \\
\end{enumerate}
\begin{example-N}
	Suppose $p(t) = 2t$. Then $e^{\int p(t) dt} = e^{\int 2t dt} = e^{t^2 + c}$. Then $e^{t^2 + c} = e^{t^2} e^c = e^{t^2}K$, for $K\in \mathbb{R}$ and constant.\\
	Then $Ke^{t^2}\big[y'+p(t)y = q(t)\big]$ is same as $e^{t^2}\big[y'+p(t)y = q(t)\big]$ as far as solutions are concerned.
\end{example-N}
\newpage
\underline{\huge Some examples:}
\begin{enumerate}[label=\protect\circled{\Roman*}]
	\item {\Large \underline{Solve $ty' - 2y = t^3e^{-2t}$}}\\
	\newline
	\underline{\large Strategy:} This is linear so we use the integrating factor $\intfac$ to solve using the 3 steps above.\\
	\underline{\large Solution:} Place the ODE in standard form.
	\begin{equation*}
		y' - \frac{2}{t}y = t^2e^{-2t}
	\end{equation*}
	This gives us $p(t) = \frac{-2}{t}$, so the int. factor is $\intfac = e^{-2 \int \frac{1}{t}dt} = e^{-2 \ln |t|} = e^{\ln t^{-2}} = \underline{t^{-2}}$
	\begin{enumerate}[label=\textbf{Step \arabic*:}]
	
	\setlength{\itemindent}{1.3in}
	\item Multiply ODE by integrating factor.
	\begin{equation*}
		= t^{-2}\bigg[y'-\frac{2}{t}y = t^2e^{-2t}\bigg]
	\end{equation*}
	\begin{equation*}
		= \underbrace{t^{-2}y'-\frac{2}{t^3}y} = e^{-2t}
	\end{equation*}
	\begin{equation*}
		= \dfrac{d}{dt}\big[t^{-2}y\big] = e^{-2t}
	\end{equation*}
	\item Integrate wrt $t$
	\begin{equation*}
		= \int \dfrac{d}{dt}\big[t^{-2}y\big] dt = \int e^{-2t} dt
	\end{equation*}
	\begin{equation*}
		\Rightarrow t^{-2}y = \frac{-e^{-2t}}{2} + K
	\end{equation*}
	\item Solve for $y(t)$
	\begin{equation*}
		\boxed{y(t) = \frac{t^2e^{-2t}}{2} + Kt^2} \tag{this solves the ODE}
	\end{equation*}
	
	\end{enumerate}
	
	\item {\Large \underline{$x' + 2tx = t^3$. Solve this.}}\\
	\newline
	\underline{\large Strategy:} Use the integrating factor on this linear ODE to integrate through to an expression for $x(t)$\\
	\underline{\large Solution:} This ODE is linear, with $p(t) = 2t$. Thus the int. factor is $\intfac = e^{\int 2tdt} = \underline{e^{t^2}}$
	
	\begin{enumerate}[label=\textbf{Step \arabic*:}]
	
	\setlength{\itemindent}{1.3in}
	\item Multiply ODE by integrating factor.
	\begin{equation*}
		e^{t^2}\big[x' + 2tx = t^3\big]
	\end{equation*}
	\begin{equation*}
		\underbrace{e^{t^2}x' + 2txe^{t^2}} = t^3e^{t^2}
	\end{equation*}
	\begin{equation*}
		\frac{d}{dt}\big[e^{t^2}x \big] = t^3e^{t^2}
	\end{equation*}
	\item Integrate w.r.t. $t$
	\begin{equation*}
			\int \frac{d}{dt}\big[e^{t^2}x \big] dt = e^{t^2}x + c_1 = \underbrace{\int t^3e^{t^2} dt}
	\end{equation*}
	\begin{center}
		Expand using by parts and substitution. $s = t^2$, $ds = 2tdt$
	\end{center}
	\begin{align*}
		\int t^3e^{t^2} dt \longrightarrow & \frac{1}{2} \int se^sds\\
		= & \frac{1}{2}e^s(s-1) + c_2\\
		= & \frac{1}{2}e^{t^2}(t^2-1) + c_2
	\end{align*}
	\begin{center}
		Combine constants to set:
	\end{center}
	\begin{equation*}
		e^{t^2}x = \frac{1}{2}e^{t^2}(t^2-1) + K
	\end{equation*}
	\item Solve for x(t) to get the general solution.
	\begin{equation*}
		\boxed{x(t) = \frac{1}{2}t^2 - \frac{1}{2} + Ke^{-t^2}} \quad 
	\end{equation*}
	\end{enumerate}
	\item {\Large Solve $\frac{dx}{ds} = \frac{x}{s} - s^2$, for $s > 0$}\\
	Here the ODE is again linear (note $s$ is the independent variable), and $p(s) = -\frac{1}{s}$, the integrating factor is then\ldots
	\begin{equation*}
		\mathlarger{\intfac = e^{-\int (\frac{1}{s})ds} = e^{-\ln s} = e^{\ln s^{-1}} = \underline{s^{-1}}}
	\end{equation*}
	\begin{enumerate}[label=\textbf{Step \arabic*:}]
	\setlength{\itemindent}{1.3in}
	\item Multiply through standard form ODE to set
	\begin{align*}
		\frac{1}{s}\bigg[\dfrac{dx}{ds} - \dfrac{x}{s} = -s^2 \bigg] \Rightarrow & \underbrace{\dfrac{1}{s} \dfrac{dx}{ds} - \dfrac{x}{s^2}} = -s\\
		& \dfrac{d}{ds}\bigg[\dfrac{1}{s} \cdot x \bigg] = -s
	\end{align*}
	\item Integrate w.r.t. $s$ to set
	\begin{equation*}
		\dfrac{1}{s}\cdot x = \int (-s)ds + c = \dfrac{-s^2}{2} + c
	\end{equation*}
	\item Solve for x(s):
	\begin{equation*}
		\boxed{x(s) = \dfrac{-s^3}{2} + cs}
	\end{equation*}
	\end{enumerate}
\end{enumerate}
