\classheader{2018-08-20}
\section*{Systems of First Order Linear Equations}
\subsection*{Introduction}
Consider the "system" of 2 1st order ODEs in 2 variables:
\begin{equation*}
	\begin{rcases}
		\frac{dx}{dt} = a_1x - b_1 xy\\
		\frac{dy}{dt} = -a_2y - b_2 xy
	\end{rcases} a_1, a_2, b_1, b_2 > 0 \text{ constants.}
\end{equation*}
Here, both $x(t)$, $y(t)$ are functions of time, and these evolution (derivatives) are intertwined (coupled).\\
Many applications appear this way. These are called the Lotka-Volterra equations: model the population size of 2 species in a closed environment (predator and prey). A solution is a set of expressions for $x(t)$ and $y(t)$ that satisfy both equations.\\
\textbf{Question:  } Say $x(t)$ and $y(t)$ represent rabbits and foxes (not necessarily respectively). Can you tell from the ODE system which is which? How?\\
Why study systems of coupled equations? @ reasons:
\begin{enumerate}[label=\protect\circled{\arabic*}]
	\item Many apps appear this way. There are many measurable quartiles all depending on a single independent variable (not like vector calculus). In general, this looks like
	\begin{gather*}
		\begin{rcases}
			\dot{x_1} = F_1(t_1x_1, \cdots, x_n)\\
			\dot{x_1} = F_2(t_1x_1, \cdots, x_n)\\
			\vdots\\
			\dot{x_n} = F_n(t_1x_1, \cdots, x_n)\\
		\end{rcases}
		\text{1st order system of ODEs}
	\end{gather*}
	where $x_1, \cdots, x_n$ are the set of n dependent variables and time $t$ is the independent variable
	\item \underline{Any} higher ODE can be transformed (rewritten) as a system of 1st order ODEs:
	\begin{equation*}
		\text{Let } \quad y^{(n)} = F(t_1y_1y', \cdots, y^{(n-1)})
	\end{equation*}
	Given the new vars:
	\begin{gather*}
		x_1 = y, \quad x_2 = y', \quad x_3 = y'', \quad \cdots, x_n = y^{(n-1)}\\
		\text{we get } \quad \dot{x_1} = x_2, \quad \dot{x_2} = x_3, \quad \dot{x_3} = x_4, \quad \cdots \dot{x_{n-1}} = x_n, \quad \dot{x_n} = F(t_1x_1, \cdots, x_n)\\
		\underline{and} \\
		\dot{x_1} = \dot{y} = y' = x_2\\
		\dot{x_2} = (y')' = y''= x_3\\
		\vdots\\
		\dot{x_n} = (y^{n-1})' = y^{(n)} = F
	\end{gather*}
\end{enumerate}
A solution to ($\star$) is a set of functions
\begin{equation*}
	x_1(t), x_2(t), \cdots, x_n(t)
\end{equation*}
If initial values are specified, we would need 
\begin{gather*}
	\begin{rcases}
		\text{1 date at each } x_i(t), i = 1, \cdots, n\\
		x_1(t_0) = x_1^0, \cdots, x_n(t_0) = x_n^0\\
		\text{This is identical to }\\
		y(t_0) = y_0, \cdots, y^{(n-1)}(t_0) = y_0^{(n-1)}
	\end{rcases} \text{n-bits of initial data}
\end{gather*}
\underline{Existence and uniqueness for 1st Order systems} (similar to that for a single eqn).
\begin{theorem-N}
	In ($\star$), let $F_1, \cdots, F_n$ \underline{and} all of
	\begin{equation*}
		\dfrac{dF_1}{dx_1}, \cdots, \dfrac{dF_1}{dx_n}, \quad \dfrac{dF_2}{dx_1}, \cdots, \dfrac{dF_2}{dx_n}, \quad \cdots, \quad \dfrac{dF_n}{dx_1}, \cdots, \dfrac{dF_n}{dx_n}, \quad 
	\end{equation*}
	\begin{center}
		(all partials w.r.t. dep. vars $x_i$ but not $t$) be continuous in a region R of the $(n+1)$-dimiension $t_x, \cdots, x_n$-space defined by $\alpha < t < \beta, \quad \alpha_1 < x_1 < \beta_1, \quad \cdots, \alpha_n < x_n, \beta_n$ 
	\end{center}
	$\Rightarrow$ on an interval $|t - t_0| < h$, there is a unique solution to ($\star$) defined on the interval and passing through p.
\end{theorem-N}
\textbf{Note:} The proof is similar to that of the 1-dimension case
\begin{definition}
	($\star$) is linear if each $F_i$ is linear in all of the $x_i$'s $i = 1,\quad, n$. Indeed if 
	\begin{equation*}
		(\ast)
		\begin{cases}
			x_1' = p_{11}(t)x_1 + \cdots + p_{1n}(t)x_n + g_1(t)\\
			x_2' = p_{21}(t)x_1 + \cdots + p_{2n}(t)x_n + g_2(t)\\
			\quad \quad \quad \vdots\\
			x_n' = p_{n1}(t)x_1 + \cdots + p_{nn}(t)x_n + g_n(t)\\
		\end{cases}
	\end{equation*}
	and \underline{homogenous} if each $g_i(t) \equiv 0$.
\end{definition}
\textbf{Note: } Solutions exist \underline{and} one unique for a linear system of IDEs (like ($\ast$) on an interval I it \underline{all} $p_{ij}(t)$ and $g_i(t)$ and continuous on I).