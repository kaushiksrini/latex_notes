\classheader{2018-07-09}
\subsection*{Solutions of Linear Homogenous Equations}
New Questions: If you found 2 solutions $y_1$, $y_2$ to $L[y] = 0$, do all solutions look like $c_1y_1 + c_2y_2$? Can there be others?\\
To study this, let's "solve" the IVP.
\begin{equation*}
	L[y] = 0, \quad y(t_0) = y_0, \quad y'(t_0) = y_0'
\end{equation*}
using the idea of a "general" solution
\begin{equation*}
	y(t) = c_1y_1(t) + c_2y_2(t)
\end{equation*}
\begin{align*}
	\overbrace{c_1y_1(t_0)}^{\text{real \#}} + \overbrace{c_2y_2(t_0)}^{\text{real \#}} = y_0\\
		c_1y_1'(t_0) + c_2y_2'(t_0) = y_0' \tag{$\star \star$}
\end{align*}
Solve the system for $c_1, c_2$ (2 eqns., 2 unknowns)\\
\redhline\\
\begin{equation*}
	\begin{split}
		c_1 = \dfrac{y_0 \cdot y_2'(t_0) - y_0' \cdot y_2(t_0)}{y_1(t_0) \cdot y_2'(t_0) - y_1'(t_0) \cdot y_2(t_0)}
	\end{split}
	\quad \quad \quad 
	\begin{split}
		c_2 = \dfrac{y_0 \cdot y_1'(t_0) - y_0' \cdot y_1(t_0)}{y_1(t_0) \cdot y_2'(t_0) - y_1'(t_0) \cdot y_2(t_0)}
	\end{split}
\end{equation*}
\textbf{Note:} Solutions to $(\star \star)$
\begin{center}
	1 solution \quad lines cross\\
	0 solutions \quad lines parallel\\
	$\infty$ solutions \quad lines the same
\end{center}
\textbf{Note:} the numerators are different for $c_1$, $c_2$ but the \underline{denominators} are \underline{the same!}\\
Rewrite the denominator as the determinant of a $2x2$ matrix whose entries are the coefficients of $(\star \star)$:
\begin{equation*}
	y_1(t_0) \cdot y_2'(t_0) - y_1'(t_0) \cdot y_2(t_0) = 
	\begin{vmatrix}
		y_1(t_0) & y_2(t_0) \\
		y_1'(t_0) & y_2' (t_0)
	\end{vmatrix}
\end{equation*}
This comes from writing $(\star \star)$ as a matrix equation:
\begin{equation*}
	\underbrace{\begin{bmatrix}
		y_1(t_0) & y_2(t_0) \\
		y_1'(t_0) & y_2' (t_0)
	\end{bmatrix}}_{A}
	\begin{bmatrix}
		c_1\\
		c_2
	\end{bmatrix}
	=
	\begin{bmatrix}
		y_0\\
		y_0'
	\end{bmatrix}
\end{equation*}
Given this matrix equation, if $\det A \neq 0$ there is a unique solution $(c_1, c_2)$\\
\redhline\\\\
In our case, in the 2 expressions for $c_1, c_2$
\begin{enumerate}[label=\protect\circled{\Roman*}]
	\item If denominator non-zero, then unique solution.
	\item if \underline{ONLY} denominator zero, then no solutions
	\item If both numerator, denominator zero, tons of solutions
\end{enumerate}
\begin{equation*}
	\text{Call} \quad W = W(y_1, y_2)(t_0) = 
	\begin{vmatrix}
		y_1(t_0) & y_2(t_0) \\
		y_1'(t_0) & y_2' (t_0)
	\end{vmatrix}
\end{equation*}
The \textbf{Wronskian determinant} of $y_1$, $y_2$ at $t_0$.
\begin{itemize}
	\item Tells you about the solutions to the IVP - $L[y] = 0$, $y(t_0) = y_0$, $y'(t_0) = y_0'$
\end{itemize}
\redhline
\begin{theorem-N}
	Suppose $y_1$, $y_2$ are 2 solutions to $l[y] = 0$ and at the initial values $y(t_0) = y_0$, $y_0'(t_0) = y_0'$ and $W(y_1, y_2)(t_0) \neq 0$
	\begin{center}
		$\Rightarrow \exists c_1, c_2 \in \mathbb{R}$ so that $y(t) = c_1 \cdot y_1(t) + c_2 \cdot y_2(t)$ solves the IVP.
	\end{center}
\end{theorem-N}
Note: This ensures that $y_1$ and $y_2$ are fundamentally "different" solutions (read: independent)
\begin{center}
	\textbf{Question:} What does this mean?
\end{center}
\begin{theorem-N}
	If $y_1$, $y_2$ both solve $L[y] = 0$, and if $\exists t_0$ where $W = W(y_1, y_2)(t_0)$, $\Rightarrow y(t) = c_1 y_1(t) + c_2 y_2(t)$ includes every solution!
\end{theorem-N}
\begin{proof}
	Let $\gamma$ be any solution to the IVP near $t_0$, where $W = W(y_1, y_2)(t_0)$. Then by \underline{theorem 1}, $c_1 y_1(t) + c_2 y_2(t)$ solves the IVP for some choice of $c_1, c_2 \in \mathbb{R}$. But by uniqueness $\gamma (t) = c_1 y_1(t) + c_2 y_2(t)$
\end{proof}
Here, given $L[y] = 0$, if you find any 2 solutions $y_1$, $y_2$ where Wronskian is somewhere non-zero, then $y(t) = c_1 y_1(t) + c_2 y_2(t)$ includes \underline{all} solutions on the entire \underline{interval} where Wronskin is non-non-zero. Called the general solution or the fundamental set of solutions to $L[y] = 0$\\
\begin{example-N}
	Suppose $y_1 = e^{\Gamma_1 t}$ and $y_2 = e^{\Gamma_2 t}$ both solve $L[y] = 0$. The Wronskin is
	\begin{equation*}
		W(y_1, y_2)(t) = 
		\begin{vmatrix}
			e^{\Gamma_1 t} & e^{\Gamma_2 t}\\
			\Gamma_1 e^{\Gamma_1 t} & \Gamma_2 e^{\Gamma_2 t}
		\end{vmatrix}
		 = (\Gamma_1 + \Gamma_2) e^{(\Gamma_1 + \Gamma_2)t}
	\end{equation*}
	Here, as long as $\Gamma_1 \neq \Gamma_2$, $W(y_1, y_2)(t) \neq 0$ anywhere on $\mathbb{R}$, and $y(t) = c_1 e^{\Gamma_1 t} + c_2 e^{\Gamma_2 t}$ is a fundamental set of solutions to $L[y'] = 0$
\end{example-N}
\redhline\\
\textbf{Example} $y_1 = \sin t$, $y_2 = \cos t$ and $W(y_1, y_2)(t) \equiv 1$\\
\textbf{Example} $y_1 = \sin t$, $y_2 = \cos (t - \frac{\pi}{2})$ and $W(y_1, y_2)(t) \equiv 0$\\
Where $W(y_1, y_2)(t) \neq 0$, we can say $y_1$, $y_2$ are linearly independent as functions.\\
\begin{definition}
	Two functions $f(x)$, $g(x)$ are called \underline{linearly dependent} \textbf{(LD)} on some open interval I if there exists 2 constants $k_1$, $k_2$ are not both 0, where
	\begin{equation*}
		k_1 f(x) + k_2 g(x) = 0
	\end{equation*}
	$\forall x \in I$. Otherwise called \underline{linearly independent} or \textbf{(LI)}
\end{definition}
\underline{Note:} If there exists one point $x \in I$ where 2 functions are \textbf{LI}, then the functions are \textbf{LI} on I.\\
\underline{Extra} The Wronskin only really depends on the ODE in a fundamental way:
\begin{theorem}
	Given any 2 solutions to $L[y] = 0$, where $p(t)$, $q(t)$ are continuous on an open interval $I$, then
	\begin{equation*}
		W(y_1, y_2) = ce^{-\int p(t) dt}
	\end{equation*}
	where $c$ depends on $y_1$, $y_2$ but not on $t$.
\end{theorem}
\underline{Notes}
\begin{enumerate}[label=\protect\circled{\Roman*}]
	\item If $y_1$, $y_2$ are \textbf{LD}, then $c=0$
	\item If $y_1$, $y_2$ are \textbf{LI}, then $W \neq 0$ on all of $I$
	\item Proof is quire interesting!
	\item LI and non Wronskin are the same thing for ODEs.
\end{enumerate}
\begin{proof}
	Since $y_1$, $y_2$ solve the ODE
	\begin{enumerate}[label=\protect\circled{\alph*}]
		\setlength\itemindent{25pt} \item $y_1'' + p(t)y_1' + q(t)y_1 = 0$
		\item $y_2'' + p(t)y_2' + q(t)y_2 = 0$
	\end{enumerate}
	Must \circled{a} by $-y_2$ and \circled{b} by $y_1$ and add (eliminating $q(t)$)
	\begin{equation*}
		\underbrace{\underbrace{y_1 y_2'' - y_2 y_1''}_{W'(y_1, y_2)} + p(t) \underbrace{y_1 y_2' - y_2 y_1'}_{W(y_1, y_2)}}_{W' + p(t)W = 0} = 0
	\end{equation*}
	is a 1st order ODE in the Wronskin $\det$ as a function of $t$. By separation of variables:
	\begin{align*}
		\frac{W'}{W} = -p(t) & \Rightarrow \ln |W| = - \int p(t) \det K\\
		& \Rightarrow W = ce^{-\int p(t) dt}
	\end{align*}
	Eitehr $W = 0$ on all of $I$ or $W \neq 0$ on all of $I$
\end{proof}
