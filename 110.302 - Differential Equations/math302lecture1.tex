\classheader{2018-06-14}

Start with $y = f(x)$, some unknown functional relation between 2 variables, where
\begin{itemize}
	\item $x$ - independent variable
	\item $y$ - dependent variable
\end{itemize}

Such an equation (or sets of them are) called a \underline{mathematical model} when the variables represent measuarble quantities in some application \textit{(usually set up to study some unknown entity of based on its relationship to something controllable x).}\\
If $y = f(x)$ is known, then we can simply study its properties (using calculus).

Often, though, we do not know $y = f(x)$, but we do have information about some properties, like derivatives, for example.

\textbf{Examples:}
\begin{enumerate}[label=\protect\circled{\arabic*}]
	\item $\frac{dx}{dy} = ky, k \in \mathbb{R}$
	\item $F = ma$ (Newton's 2nd Law of motion )
	\item $f'(x) = x - e^{x/2}$ (restraint of: Find $\int (x - e^{x/2}) d\textit{x}$)
	\item $\frac{d^2\theta}{dt^{2}} + \frac{g}{L} \sin \theta \rightarrow \textbf{The Pendulum}$
\end{enumerate}

The above are mathematical models whhere the actual function is only known implicitly...

\section*{Ordinary Differential Equations}

\begin{definition-N}
	An \underline{Ordinary Differential Equation} \textbf{(ODE)} is an equation involving an unknown function between two entities and some of its derivatives\\
\underline{Note:} "Ordinary" means that the unknown function is a function of one independent variable
\end{definition-N}

\begin{example-N}
	\textbf{The Heat Equation (in 3-space)}\\
	\center
	$\customderiv{u}{t} = \alpha(\dfrac{\partial^2u}{\partial x^2} + \dfrac{\partial^2 u}{\partial y^2} + \dfrac{\partial^2 u}{\partial z^2})$
	
	is a \underline{partial} differential equation since $'u'$ is a function of more than 1-independent variable.
\end{example-N}

\begin{definition-N}
	The \underline{order} of an ODE is the same as the order of the highest derivative that appears in the equation:
\end{definition-N}

\begin{definition-N}
	The \underline{general form} of an nth order ODE is 
	\begin{equation*}
		\boxed{F(x, y, y', y'', ..., y^{(n)}) = 0}
	\end{equation*}
	\begin{center}
		x-independent variable\\
	y-dependent variable $y = f(x)$\\
	$y^{(i)}$ is the ith derivative of $y = f(x)$\\
	$F$ is some expression in $x, y, y', y'', ..., y^{(n)}$\\
	\end{center}	
	\underline{Note:} \underline{Sometimes}, we can solve for the highest derivative
	\begin{equation*}
		\boxed{y^{(n)} = G(x, y, y', y'', ..., y^{(n-1)})}
	\end{equation*}
	But this is not always possible, since $y^{(5)} + \sin y^{(5)} = y^{(3)}$ cannot be simplified
\end{definition-N} 
\redhline
\begin{definition-N}
	A function $f(\underbrace{x_1, x_2, ..., x_n}_{\vec{x}}, \underbrace{y_1, ..., y_n}_{\vec{y}})$ is \underline{linear} in the variables $y_1, ..., y_n$ if \\$f(x_1, x_2, ..., x_n, y_1, ..., y_n) = G_0(\vec{x}) + \sum_{i = 1}^m G_i(\vec{x})y_i$ where the $G_i(x), i=0,...,m$ are arbitrary.\\\\
	\underline{Notes:} \circled{1} For an ODE to be \underline{linear}, it must be linear in $y, y', ... , y^{(n)}$, and can be written as:
	\begin{equation*}
		\boxed{G_n(x) y^{(n)} + \cdots + G_1 (\vec{x})\cdot y' + G_0(\vec{x})\cdot y = g(x)}
	\end{equation*}
\end{definition-N}

\textbf{Examples:}
	\begin{enumerate}[label=\protect\circled{\arabic*}]
		\item $(sinx)y' + (lnx)y = tan (e^x)$ is a linear \underline{first} order ODE.
		\item $y'' + xy' + siny = 0$ is \underline{NOT} linear.
		\item $y''y + y' = 0$ is \underline{NOT} linear.
	\end{enumerate}
\redhline

Suppose $y' = f(t, y)$ is a \underline{first} order linear ODE. Then thehere exist functions $p(t), q(t)$ so that $f(t, q) = -p(t)y + q(t)$, and the ODE can be written as
\begin{equation*}
\tcbhighmath[drop fuzzy shadow]{y' \pm p(t)y = q(t)} \tag{$\star$}
\end{equation*}
This form will be very important to understanding how to study this type of ODE.

\begin{example-N}
	Given $y'\sin x  + y\ln x = \tan e^x$, identify $p(t)$\\
	\underline{Solution:} Divide by $sinx$ to set,
	\begin{equation*}
		y' + \dfrac{\ln x}{\sin x}y = \dfrac{\tan e^x}{\sin x}. \quad \boxed{p(t) = \dfrac{\ln x}{\sin x}}
	\end{equation*}
\end{example-N}

\begin{definition-N}
	A \underline{solution} to $(\star)$ on $I = (a,b) \subset \mathbb{R}$ is any function $y = f(x)$ that satisfies that equation $(\star)$.
	Sometimes a solution is only known \underline{implicitly}
\end{definition-N}

Let's play this out and see just how the integrating factor is helpful.
