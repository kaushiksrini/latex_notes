\classheader{2018-08-15}
\section*{Higher Order Linear Equations}
\subsection*{General Theory of nth Order Linear Equation}
The n-th order version. of a linear ODE is
\begin{equation*}
	a_n(t)y^{(n)} + a_{n-1}(t)y^{(n-1)} + \cdots + a_1(t)y^{(1)} + 	a_n(0)y = G(t)
\end{equation*}
which can also be written like an operator
\begin{equation*}
	L[y] = y^{(n)} + p_1(t)y^{(n-1)} + p_2(t) y^{(n-2)} + \cdots + p_{n-1}(t)y^{(1)} + p_n(t)y = g(t) \tag{$\star$}
\end{equation*}
where we divide each of the coefficients in the top description by $a_n(t)$ (example $p(t) = \frac{a_{n-1}(t)}{a_n(t)}$)\\\\
The theory generalizes in the obvious ways:
\begin{enumerate}[label=\protect\circled{\Roman*}]
	\item If the ODE is an IVP, then we will need $n$ pieces of information to completely determine a solution (think $n$ integration to get solution creating an $n$-parameter family of functions): $y(t_0) = y_0$, $y^1(t_0) = y_0^1$, $\cdots$, $y^{(n-2)}(t_0) = y_0^{(n-2)}$, $y^{(n-1)}(t_0) = y_0^{(n-1)}$
	\item \textbf{Theorem.}	\textit{(Existence and Uniqueness)\\
		In ($\star$) if $p_1, \cdots, p_n$ gave all continuous on some common interval I, then there exists a unique solution to ($\star$) passing through any set of initial values $t_0 \in I$.}
	\item Superposition Holds: if $y_1(t)$ and $y_2(t)$ both solve $L[y] = 0$ (homogenous) where $L[y]$ corresponds to the nth order ODE $(\star)$ then $c_1y_1(t) + c_2y_2(t)$ is also a solution.
	\item Given n solutions to $y_1, \cdots, y_n$ to an nth order homogenous $L[y] = 0$, if 
	\begin{equation*}
		W(y_1, \cdots, y_n) (t) = 
		\begin{vmatrix}
			y_1(t) & y_2(t) & \cdots & y_n(t)\\
			y_1^1(t) & y_2^1(t) & \cdots & y_n^1(t)\\
			\vdots & \vdots & \ddots & \vdots\\
			y_1^{(n-1)}(t) & y_2^{(n-1)}(t) & \cdots & y_n^{(n-1)}(t)\\
		\end{vmatrix}
	\end{equation*}
	is nonzero at $y \in I$, then every solution to the ODE is a linear combination of $y_1$, $\cdots$, $y_n$. In this case, a fundamental set of solutions then is $y(t) = c_1y_1(t) + \cdots + c_ny_n(t)$
	\item In fact, it can be shown that for any choice of $y_1$, $\cdots$, $y_n$ solutions to $(\star)$, 
	\begin{equation*}
		W(y_1, \cdots, y_n) = ce^{-\int p_1(t)dt}
	\end{equation*}
	and is either always 0 on I where $p_1(t)$ is continuous or never 0 on $I$.
	\item If in $(\star)$, $g(t) \neq 0$, then a general solution is the same as thjat of a 2nd order non-homogenous ODE:
	\begin{equation*}
		y(t) = \underbrace{c_1y_1(t) + \cdots + c_ny_n(t)}_{\text{fund. set of the solns. of} L[y] = 0} + \overbrace{\Ylines(t)}^{\text{any particular soln. to } L[y] = g(t)}
	\end{equation*} 
\end{enumerate}