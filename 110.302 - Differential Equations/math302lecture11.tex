\classheader{2018-07-09}
Before talking about the cases where roots of the characteristic equation are the same or not real, lets return to the more general linear 2nd order homogenous ODE
\begin{equation*}
	y'' + p(t)y' + q(t)y = 0
\end{equation*}
To study this, form the operator (an operator is a function whose domain and range are functions).
\begin{equation*}
	L[\gamma] = \gamma'' + p(t)\gamma' + q(t) \gamma
\end{equation*}
This operator is defined for all $c^2$ functions $y(t)$ on an interval like $\alpha < t < \beta$, where $\alpha$ may be a number or $-\infty$, and $\beta$ may be a number or $\infty$.\\
\underline{\large Notes}
\begin{enumerate}[label=\protect\circled{\Roman*}]
	\item Can also write
	\begin{equation*}
		L = \dfrac{d^2}{dt^2} + p\dfrac{d}{dt} + q
	\end{equation*}
	\item An operator $L[\gamma]$ is \underline{linear} if $L[c_1\gamma_1 + c_2\gamma_2] = c_1L[\gamma_1] + c_2L[\gamma_2]$
	\begin{claim}
		$L[\gamma] = \gamma'' + p(t)\gamma' + q(t)$ is linear as an operator
	\end{claim}
	\begin{proof}
	\begin{align*}
		L[c_1\gamma_1 + c_2\gamma_2] & = \dfrac{d^2}{dt^2}\big[ c_1\gamma_1 + c_2\gamma_2\big] + p(t) \dfrac{d}{dt}\big[ c_1\gamma_1 + c_2\gamma_2 \big] +q(t)(c_1\gamma_1 + c_2\gamma_2)\\
		& = c_1\gamma_1'' + c_2\gamma_2'' + p(t)(c_1\gamma_1' + c_2\gamma_2') + q(t)(c_1\gamma_1'')\\
		& = c_1(\gamma_1'' + p(t)\gamma_1' + q(t)\gamma_1) + c2(\gamma_2'' + p(t)\gamma_2' + q(t)\gamma_2)\\
		& = c_1L[\gamma_1] + c_2L[\gamma_2]
	\end{align*}
	\end{proof}
\end{enumerate}
\underline{\large Fact:} The homogenous 2nd order linear ODE
\begin{equation*}
	y'' + p(t)y' + q(t)y = 0
\end{equation*}
is solved by \underline{any} function $y(t)$, where $L[y(t)] = 0$\\
\redhline
\begin{center}
	\Large \textbf{2 theorems on linear, $2^{nd}$ order ODE}
\end{center}
\redhline
\begin{enumerate}[label=\protect\circled{\Roman*}]
	\item Existence of Uniqueness
	\begin{theorem}
		The IVP $y'' + p(t)y' + q(t)y = g(t)$, $y(t_0) = y_0$, $y'(t_0) = y_0'$, where $p$, $q$, and $g$ are continuous on an open interval I containing $t_0$, has a unique solution $y(t)$ defined and twice differentiable on I.
	\end{theorem}
	\underline{\textbf{Note:}} Here, I can be taken to be the largest interval containing $t_0$, where $p$, $q$, and $g$ are all simultaneously continuous.
	\item Superposition
	\begin{theorem}
		If $y_1(t)$, $y_2(t)$ are 2 solutions to $L[y] = 0$, then so is $c_1\gamma_1 + c_2\gamma_2$ for all $c_1, c_2 \in \mathbb{R}$
	\end{theorem}
\end{enumerate}
