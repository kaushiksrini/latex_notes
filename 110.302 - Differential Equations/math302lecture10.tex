\classheader{2018-07-08}
I will relegate the discussion of Section of 2.8 to a worksheet posted. The theory can be deep (through very interesting). But the main takeaway is its usefulness in...
\begin{enumerate}[label=\protect\circled{\Roman*}]
	\item seeing where a first order ODE is "nice"
	\item Understanding the intricacies of the theory even at this early stage.
\end{enumerate}
\redhline
\section*{Second Order Linear Equations}
\subsection*{Homogeneous Equations with Constant Coefficients}
A general form of a 2nd order ODE is, for some function $f$
\begin{equation*}
	\tcbhighmath[drop fuzzy shadow]{y'' = f(t, y, y')} \tag{$\ast \ast$}
\end{equation*}
\begin{definition-N}
	A 2nd order ODE is called \underline{linear} if it can be written
	\begin{equation*}
		y'' + p(t)y' + q(t)y = g(t)
	\end{equation*}
	\begin{center}
		(so that $f(t, y, y') = g(t) - p(t)y' - q(t)$ in ($\ast$))\\
		($\ast \ast$) (actually $P(t)y'' + Q(t)y' + R(t)y = G(t)$...)
	\end{center}
	\underline{Notes}
	\begin{enumerate}[label=\protect\circled{\Roman*}]
		\item $f$ needs to be linear in both $y$ and $y'$
		\item If ODE is not linear, it is called non-linear.
	\end{enumerate}
\end{definition-N}
\begin{definition-N}
	If $g(t) \equiv 0$, then a linear ODE is called \underline{homogeneous}.
\end{definition-N}
\begin{definition-N}
	An IVP with a 2nd order ODE continuous 2 pieces of initial data, usually $y(t_0) = y_0$, $y'(t_0) = y_0'$
\end{definition-N}
\underline{Question:} Why?\\
In General, it is heard or impossible to solve a 2nd order ODE. Even linear is very difficult in general!\\
\redhline\\\\
One type that is solvable: Constant coefficients. Let ($\ast \ast$) here $P(t) \equiv a$, $Q(t) \equiv b$, $R(t) \equiv c$, and suppose $G(t) \equiv 0$ (homogenous).
\begin{equation*}
	\text{Then ODE is } \quad \tcbhighmath{ay'' + by' + cy = 0} \tag{$\star$}
\end{equation*}
\textbf{Q: } First think, what kinds of functions would possibly be solutions to their kind of ODE?
\begin{multicols}{2}
	\begin{itemize}
	\item Polynomials? Prove Functions?
	\item Trig functions?
	\item Exponentials?
	\item Logarithms?
	\end{itemize}
\end{multicols}
\begin{example-N}
	Suppose $a =1$, $b = 0$, $c = -1$, Then ($\ast \ast$) is $y'' - y = 0$, or $y'' = y$\\
	\underline{\large Solutions?}\\ Here $y(t) = e^t$ and $y(t) = e^{-t}$ both solve $y'' - y = 0$ How about $e^{t} + e^{-t}$? $2e^t - 3e^{-t}$? Here $y(t) = c_1e^t + c_2e^{-t}$ is a solution for any choice of $c_1, c_2 \in \mathbb{R}$\\
	What if IVP was $y''-y=0$, $y(0)=3$, $y(0) = 4$?
	\begin{equation*}
		y(0) = 3 = c_1 e^0 + c_2 e^{-0} = c_1 + c_2
	\end{equation*}
	\begin{equation*}
		\underbrace{y(0) = 4 = c_1 e^0 - c_2 e^{-0} = c_1 - c_2}_{\begin{split}c_1 = \frac{7}{2}\\ c_2 = -\frac{1}{2} \end{split} \quad \begin{cases}
			3 = c_1 + c_2\\ 4 = c_1 - c_2
		\end{cases}}
	\end{equation*}
	And the particular solution to IVP is 
	\begin{equation*}
		y(t) = \frac{7}{2}e^t - \frac{1}{2} e^{-t}
	\end{equation*}
\end{example-N}
\begin{example-N}
	$2y'' + 8y' -10y = 0$\\
	Here $y(t) = e^t$ and $y(t) = e^{-5t}$ both work! Check this... \underline{AND} so does $y(t) = c_1 e^t + c_2 e^{-5t}$. \underline{Is there a pattern?}
\end{example-N}
\redhline\\
For $ay'' + by' + cy = 0$, assume the solution is exponential (is this a good idea?) and looks like $y(t) = e^{\Gamma t}$, where $\Gamma$ is an unknown parameter.\\
Then substituting $y(t)$ and its derivatives into the ODE, we set
\begin{equation*}
	a\Gamma^2e^{\Gamma t} + b\Gamma e^{\Gamma t} + ce^{\Gamma t} = 0 \quad \textit{or} \quad a\Gamma^2 + b\Gamma + c = 0
\end{equation*}
Any valid values for $\Gamma$ must satisfy $\Gamma = \frac{-b \pm \sqrt{b^2 - 4ac}}{2a}$. \underline{Recognize this?}\\\\
For a 2nd order homogenous ODE with constant coefficients ($\star$) is called the \underline{characteristic equation}
\underline{\large \textbf{Important:}} When the two roots $\sqrt{1}$, $\sqrt{2}$ of the char. eqn. one \underline{real} and \underline{distinct}, then the general solution to the ODE is
\begin{equation*}
	\tcbhighmath[drop fuzzy shadow]{y(t) = c_1 e^{\Gamma_1 t} + c_2 e^{\Gamma_2 t}}
\end{equation*}
\begin{center}
	(we will need to make sure this is the case later)
\end{center}
\begin{example} $y''-y = 0$. Here $a = 1$, $b=0$, $c=1$, and characteristic equation is $\Gamma^2 - 1 = 0$, with roots $\Gamma = -1, 1$. General solution is 
\begin{equation*}
	y(t) = c_1e^t + c_2e^{-t}
\end{equation*}
\end{example}
\begin{example}
Characteristic equation is $2\Gamma^2 + 8\Gamma -10 = 0$ which factors to $(2\Gamma - 2)(\Gamma + 5) = 0$, so $\Gamma = 1, -5$.
\begin{equation*}
		y(t) = c_1e^t + c_2e^{-5t}	
\end{equation*}	
\end{example}

