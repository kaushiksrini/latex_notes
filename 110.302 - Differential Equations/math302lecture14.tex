\classheader{2018-07-10}
\subsection*{Repeated Roots; Reduction of Order}
\underline{Reduction of order} - Using one solution to an nth-order ODE to create on $(n-1)$th order ODE.\\
Suppose ($\star$) $y'' + p(t)y' + q(t)y = 0$ has $y_1(t)$ as a non-zero solution. If you can find an independent second solution, you are done.\\
\underline{Guess:} Assume second solution has the form $y_2(t) = v(t)y_1(t)$ for some function $v(t)$.\\
Why? you will see.\\
\underline{Good: } Try to solve the $v(t)$.\\
Now if $y_2(t)$ solves the ODE, then
\begin{align*}
	y_2'(t) = \dfrac{d}{dt}[v(t)y_1(t)] = v'(t)y_1(t) + v(t)y_1(t)\\
	y_2''(t) = v''(t)y_1(t) + \underbrace{v'(t)y_1'(t) + v'(t)y_1'(t)}_{2v'(t)y_1'(t)} + v(t)y_1''(t)
\end{align*}
Substitute this back into the original ODE: 
\begin{equation*}
	\underbrace{(v''y_1 + 2v'y_1' + vy'')}_{y_2''} + p\underbrace{(v'y_1 + vy_1')}_{y_2'} + q \underbrace{vy_1}_{y_2} = 0
\end{equation*}
Recall in terms of $v$ and derivatives:
\begin{equation*}
	y_1v'' + (2y_1' + py_1)v' + \underbrace{(y_1'' + py_1' + qy_1)}_{=0}v = 0 \tag{we guess $y_2 = vy_1$}
\end{equation*}
We left with:
\begin{equation*}
	y_1v'' + (2y_1' + py)v' = 0 \tag{$\star$}
\end{equation*}
This is a 2nd order ODE in $v(t)$. But this is a 1st order ODE in $v'(t)$!!
\begin{center}
	\boxed{\text{\Large Solve for $v'(t)$. Then integrate to set $v(t)$.}}
\end{center}
{\large \underline{Notes:}}
\begin{enumerate}[label=\protect\circled{\Roman*}]
	\item Thinking of ($\star$) is a 1st order ODE in $v'(t)$ is called \underline{reducing the order}.
	\item If the coefficient of $v(t)$ weren't $0$, then we cannot do this! Since the lowest order derivative is $v'$, we can.
	\item For independence,
	\begin{align*}
		W(y_1, vy_1) & = 
		\begin{vmatrix}
		y_1 & vy_1 \\
		y_1' & v'y_1 + vy_1' \\	
		\end{vmatrix}\\
		& = v_1y_1^2 + \underbrace{vy_1y_1' - vy_1y_1'}_{\text{cancel}} = v_1y_1^2
	\end{align*}
\end{enumerate}
As long as $v_1' \neq 0$, (so $v_1(t)$ is not a constant) $y_2 = v_1y_1$ will be independent of $y$.
\begin{example-N}
	$y_1(t) = \frac{1}{t}$ solves $t^2y'' + cty' + y = 0$ on the interval $t > 0$. Find the set of fundamental solutions.\\
	\redhline
	\begin{center}
		\underline{Strategy:} Use reduction of order.
	\end{center}
	\redhline\\
	\underline{Solution:} Both $p(t) = \frac{3}{t}$, $q(t) = \frac{1}{t^2}$, are $C^0$ on $(0, \infty)$. Assume $y_2(t) = v(t)y_1(t) = \frac{v(t)}{t}$\\
	Then $v(t)$ solves
	\begin{align*}
		y_1v'' + (2y_1' + py_1)v' = 0 \text{, or }\\
		\frac{1}{t}v'' + (2(-\frac{1}{t^2}) + (\frac{3}{t})(\frac{1}{t}))v' = 0\text{, or }\\
		\frac{v''}{t} + \frac{v'}{t^2} = 0 \Rightarrow tv'' + v = \dfrac{d}{dt}[tv'] = 0 
	\end{align*}
	which implies $tv' = c$, a constant, or $v'(t) = \frac{c}{t}$, or $v(t) = c_1\ln t + c_2$ on $t=0$
	\begin{equation*}
		\text{Hence} \quad \quad y_2(t) = \frac{v(t)}{t} = \frac{c_1 \ln t + c_2}{t}
	\end{equation*}
\end{example-N} 
{\large \underline{Questions to ask}}
\begin{enumerate}[label=\protect\circled{\Roman*}]
	\item Does $y_2(t)$ actually solve the original ODE?
	\begin{equation*}
		y_2(t) = \frac{c_1 \ln t + c_2}{t}, \quad y_2'(t) = c_1 \bigg( \frac{1 - \ln t}{t^2} \bigg) - \frac{c_2}{t^2}, \quad y_2''(t) = c_1 \bigg(\frac{2 \ln t - 3}{t^3} \bigg) + \frac{2c_2}{t^3}
	\end{equation*}
	Here
	\begin{align*}
		t^2y'' + 3ty' + y = 0 = t\bigg(c_1\bigg(\frac{2 \ln t - 3}{t^3}\bigg) - \frac{2c_2}{t^3}\bigg) + 3t\bigg(c_1  \big(\frac{1 - \ln t}{t^2} - \frac{c_2}{t^2}\big)\bigg) + \frac{c_1 \ln t + c_2}{t}\\
		= c_1\bigg(\frac{2 \ln t - 3}{t} \bigg) - \frac{2c_2}{t} + 3c_1(\frac{1 - \ln t}{t}) - \frac{3c_2}{t} + c_1 \frac{\ln t}{t} + \frac{c_2}{t} = 0 
	\end{align*}
	\item Notice that $y_1 = \frac{1}{t}$ appears as a summed in $y_2 = c_1 \frac{\ln t}{t} + \frac{c_2}{t}$. By Superposition, it is not really needed. Check independence of $y_1$, $y_2$.
\end{enumerate}
Hence fundamental set of solutions is 
\begin{align*}
	y(t) & = \text{(constant)}y_1 + \text{(constant)}y_2\\
	& = \text{(constant)} \frac{1}{t} + \text{(constant)} \bigg(c_1 \frac{\ln t}{t} + c_2 \big(\frac{1}{t}\big)\bigg)
\end{align*}
Combine constants to get
\begin{equation*}
	\tcbhighmath[drop fuzzy shadow]{y(t) = \frac{K_1}{t} + K_2 \frac{\ln t}{t}} \quad \quad \text{as the general solution.}
\end{equation*}
\underline{Application} Given $ay'' + by' + cy = 0$.\\
Suppose characteristic equation has only 1 real solution.
\begin{equation*}
	\text{Then} \quad \Gamma = \frac{-b \pm \sqrt{b^2 - 4ac}}{2a} = \frac{-b}{2a}
\end{equation*}
Here $y_1(t) = e^{-\frac{b}{2a}t}$ solves the ODE, but this is the only exponential function that does. To find another function \underline{reduce the order:}\\\\
Assume $y_2 = v(t) e^{-\frac{b}{2a}t}$, where $v(t)$ solves
\begin{equation*}
	v''y + (2y_1' + py_1)v_1 = 0, \quad \text{or} \quad e^{-\frac{b}{2a}t}v'' + \underbrace{\bigg( 2(\frac{b}{2a}t e^{-\frac{b}{2a}t}) + \frac{b}{a} \frac{b}{2a}t\bigg)}_{0}v' = 0 
\end{equation*}
\begin{equation*}
	\Rightarrow e^{-\frac{b}{2a}t}v'' = 0 \Rightarrow v'' = 0 \Rightarrow v(t) = K_1 t + K_2
\end{equation*}
\begin{center}
	So $y_2(t) = (K_1 t + K_2) e^{-\frac{b}{2a}t}$
\end{center}
\begin{exercise}
Calculate $W(y_1, y_2)$ here!\\
Hence  $y_1(t), y_2(t)$ form a fundamental set of solutions,
\begin{equation*}
	y(t) = \text{(constant)} e^{-\frac{b}{2a}t} + \text{(constant)}(K_1 + K_2) e^{-\frac{b}{2a}t} 
\end{equation*}
\begin{equation*}
	\tcbhighmath[drop fuzzy shadow]{y(t) = C_1 e^{-\frac{b}{2a}t} + C_2 e^{-\frac{b}{2a}t}}
\end{equation*}
\end{exercise}
\redhline
\begin{example-N}
	Solve $25y'' - 20y' + 4y = 0, \quad y(0) = 5, y'(0) = \frac{3}{2}$\\
	\underline{Solution}: Here discriminant $b^2 - 4ac = 400 - 400 = 0$\\
	\begin{center}
	Hence $\Gamma_1 = \Gamma_2 = \Gamma = -\frac{b}{2a} = \frac{2}{5}$\\
	Hence fundamental set of solutions is:
	\begin{equation*}
		y(t) = C_1 e^{\frac{2}{5}t} + C_2 e^{\frac{2}{5}t}
	\end{equation*}
	As for the particular solution:
	\begin{equation*}
		y(0) = C_1 e^0 + C_2 e^0 = 5 = C_1
	\end{equation*}
	So $y(t) = C_1 e^{\frac{2}{5}t} + C_2 e^{\frac{2}{5}t}$\\
	And $y(t) = 5 (\frac{2}{5}) e^{\frac{2}{5}t} + C_2 e^{\frac{2}{5}t} + \frac{2C_2}{5}t e^{\frac{2}{5}t} \Bigg|_{t = 0}$\\
	$ = 2 + C_2 = \frac{3}{2} \Rightarrow C_2 = \frac{-1}{2}$
	\end{center}
	\begin{equation*}
		\text{Solution is} \quad \tcbhighmath[drop fuzzy shadow]{y(t) = 5 e^{\frac{2}{5}t} - \frac{t}{2} e^{\frac{2}{5}t}}
	\end{equation*}
\end{example-N}

