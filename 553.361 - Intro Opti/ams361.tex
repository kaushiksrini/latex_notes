%Math template by Kaushik Srinivasan

\documentclass[11pt]{article}




%----------%
%  Basics  %
%----------%


%  Specfies basic information.
%  In the metadata section of the preamble, you can specify the subject and a list of keywords for the PDF.
%
\newcommand{\coursetitle}{AMS 553.361 - Introduction to Optimization}
\newcommand{\lecturer}{Donniell Fishkind}
\newcommand{\notetaker}{Kaushik Srinivasan}
\newcommand{\notetakersemail}{ksriniv4@jhu.edu}
\newcommand{\courseterm}{Fall 2018}
\newcommand{\institution}{Johns Hopkins University}


%  array provides more column styles for the tabular and array environments.
%  (http://ctan.org/pkg/array)
%
%  parskip sets block paragraphs as the default, instead of indentation.
%  (http://www.ctan.org/pkg/parskip)
%
\usepackage[margin=1in]{geometry}
\usepackage{amsmath,amssymb,amsthm,amsfonts,array,parskip}


%  Allows equation, align, gather, etc. environments to split across pages.
\allowdisplaybreaks


%  Sets date formatting to the ISO 8601 standard, YYYY-MM-DD.
\usepackage{datetime} \renewcommand{\dateseparator}{-} \yyyymmdddate


%---------%
%  Fonts  %
%---------%


%  Defines \cal for standard calligraphy, \eucal for Euler calligraphy, and \frak for Fraktur.
\usepackage{eucal}  \let\eucal\mathcal  \let\cal\CMcal  \renewcommand{\frak}{\mathfrak}


%  Removes ligatures (e.g. the connection ordinarily made between the two f's in "differentiable").
%\usepackage{microtype} \DisableLigatures{encoding=*,family=*}


%  Removes extra space after periods.




%-------------------------------%
%  Environments and Sectioning  %
%-------------------------------%


%  Defines some standard theorem environments, in both numbered and non-numbered versions. The numbering of each enviroment will be reset for each lecture.
\newcounter{lecture}       \setcounter{lecture}{0}
\newcounter{tN}[lecture]   \newcounter{dN}[lecture]
\newcounter{lN}[lecture]   \newcounter{rN}[lecture]
\newcounter{cN}[lecture]   \newcounter{eN}[lecture]
\newcounter{pN}[lecture]

\newtheorem*{theorem}{Theorem}          \newtheorem{theorem-N}[tN]{Theorem}
\newtheorem*{lemma}{Lemma}              \newtheorem{lemma-N}[lN]{Lemma}
\newtheorem*{corollary}{Corollary}      \newtheorem{corollary-N}[cN]{Corollary}
\newtheorem*{proposition}{Proposition}  \newtheorem{proposition-N}[pN]{Proposition}

\theoremstyle{definition}
\newtheorem*{definition}{Definition}    \newtheorem{definition-N}[dN]{Definition}
\newtheorem*{remark}{Remark}            \newtheorem{remark-N}[rN]{Remark}
\newtheorem*{example}{Example}          \newtheorem{example-N}[eN]{Example}
\newtheorem*{claim}{Claim}          \newtheorem{claim-N}[cN]{Claim}

\newtheorem*{note}{\emph{Note}}

% Modifies the spacing above theorem environments, which is messed up during the parskip package

\makeatletter \def\thm@space@setup{\thm@preskip=\parskip \thm@postskip=0pt} \makeatother

%Modifies the spacing above the proof environment

\makeatletter \renewenvironment{proof}[1][\proofname]{\pushQED{\qed}\normalfont
\partopsep=\z@skip \topsep=\z@skip \trivlist \item[\hskip\labelsep\itshape #1\@addpunct{.}]
\ignorespaces}{\popQED\endtrivlist\@endpefalse} \makeatother

\renewcommand{\qedsymbol}{\rule{0.7em}{0.7em}}

%Removes extra space before and after section headings

\usepackage[compact]{titlesec}

%% PICTURES AND DIAGRAMS


\usepackage[usenames, dvipsnames]{xcolor}
\definecolor{myred}{rgb}{0.9,0.2,0.2}
\definecolor{mygreen}{rgb}{0.2,0.6,0.2}
\definecolor{myblue}{rgb}{0.2,0.2,0.8}
\usepackage{caption}
\usepackage{arydshln}
\usepackage{mathtools}

%graphicx provides advanced graphics options

\usepackage{graphicx}
\graphicspath{ {./images/} }

% tikz is for drawing all sorts of pictures and diagrams

\usepackage{tikz}
\usetikzlibrary{decorations.text}
\newcommand{\tikzmark}[2]{%
    \tikz[remember picture,baseline=-2pt]
    \node[circle,red,draw,text=black,anchor=center,inner sep=1pt] (#1) {$#2$};}
\usepackage{tikz-cd}
\usepackage{pgf, pgfplots}
\usetikzlibrary{arrows,calc,decorations,decorations.markings,fadings,positioning,patterns,shapes}
\tikzset{>=latex}
\tikzstyle{mypoint}=[inner sep=0pt,outer sep=0pt,minimum size=5pt,fill,circle]
% for circled commands
\usepackage{enumitem}
%\begin{enumerate[label=\protect\circled{\arabic*}]
\newcommand*\circled[1]{\tikz[baseline=(char.base)]{\node[shape=circle,draw,inner sep=2pt] (char) {#1};}}

% HORIZONTAL LINES

\newcommand{\redhline}{\vspace{-0.1in}{\color{myred}{\rule{\textwidth}{0.02in}}}}


%------------------------%
%  Commands and Symbols  %
%------------------------%


%  Creates commands by running over a comma-separated list. For example,
%
%     \forcsvlist{\define{\newcommand}{\textbf}{bold}}{A,B}
%
%  would create
%
%     \newcommand{\boldA}{\textbf{A}}    \newcommand{\boldB}{\textbf{B}}
%
%  (http://tex.stackexchange.com/a/5776/20882)
%
\usepackage{etoolbox}
\newcommand{\define}[4]{\expandafter#1\csname#3#4\endcsname{#2{#4}}}
\forcsvlist{\define{\DeclareMathOperator}{}{}}{im,coker,rad,nil,Ann,Ass,codim,Spec,mSpec,diam,ord,Supp,supp,disc,Ob,vol,rank,Sym,Alt,Ind}
\forcsvlist{\define{\newcommand}{\mathrm}{}}{Hom,Mor,id,GL,SL,SO,SU,U,M,Mat,Ext,Tor,Res,Cor,Inf,End,Irr,Aut,Gal,lcm,tr,sign,triv,diag,Map,op,ev,act,alg,sep,unr,nr,ab}

%  Creates commands for some names of categories in the sans-serif font.
\forcsvlist{\define{\newcommand}{\mathsf}{}}{Set,Grp,Ab,CRing,Mod,Vect,Cat,Top,PreSh,Sh,Sch,Nat,Fun,Diff}

%  Creates commands for some blackboard bold letters.
\forcsvlist{\define{\newcommand}{\mathbb}{}}{N,Z,Q,R,C,F,G,T,A,B,D}


%  Saves the section symbol, paragraph symbol, Hungarian accent, and Scandanavian O in the macros \SS, \PP, \HH, and \OO, then redefines \S, \P, \H, and \O to be the corresponding blackboard bold letters.
%
\let\SS\S  \let\PP\P  \let\HH\H  \let\OO\O
\forcsvlist{\define{\renewcommand}{\mathbb}{}}{S,P,H,O}


%  latexsym defines some alternative versions of amssymb symbols.
%  (http://www.bakoma-tex.com/doc/latex/base/latexsym.pdf)
%
\usepackage{latexsym}

\usepackage{accents}
\newcommand{\Ylines}{\underaccent{\bar}{\bar{Y}}}
\newcommand{\Xlines}{\underaccent{\bar}{\bar{X}}}
\newcommand{\Lim}[1]{\raisebox{0.5ex}{\scalebox{0.8}{$\displaystyle \lim_{#1}\;$}}}
%  Defines a copyright symbol that is a bit nicer than the built-in one.
\newcommand{\mycopyrightsymbol}{\raisebox{-0.3ex}{\tikz{\node[inner sep=0pt,outer sep=0pt] at (0,0) {\textsc{c}};\draw (0,0) circle (0.18);}}}


%  Defines commands for real and complex projective space.
\newcommand{\RP}{\mathbb{R}\mathrm{P}}  \newcommand{\CP}{\mathbb{C}\mathrm{P}}


%  Defines a bordered matrix with square bracket delimiters instead of parentheses.
%  (http://tex.stackexchange.com/questions/55054)
%
\let\bbordermatrix\bordermatrix
\patchcmd{\bbordermatrix}{8.75}{4.75}{}{}
\patchcmd{\bbordermatrix}{\left(}{\left[}{}{}
\patchcmd{\bbordermatrix}{\right)}{\right]}{}{}

%% MATH STUFF

\usepackage{relsize}
\usepackage{cancel}

%Box Colour (https://tex.stackexchange.com/questions/122945/coloured-shadowed-boxes-around-equations)
\usepackage[skins,theorems]{tcolorbox}
\tcbset{highlight math style={enhanced, colframe=red,colback=white,arc=0pt,boxrule=1pt}}
  


% Math Symbol Aliases

\newcommand{\bs}{\backslash}
\newcommand{\subeq}{\subseteq}
\newcommand{\supeq}{\supseteq}
\newcommand{\subneq}{\subsetneq}
\newcommand{\supneq}{\supsetneq}
\newcommand{\derivativex}{\dfrac{\partial}{\partial x}}
\newcommand{\customderiv}[2]{\dfrac{\partial #1}{\partial #2}}
\newcommand{\rn}{\mathbb{R}^n}
\newcommand{\re}{\mathbb{R}}
\newcommand{\sss}{\mathcal{S}}


% Accents

\newcommand{\openset}{\,\open{\subeq}\,}
\newcommand{\open}[1]{\mathring{#1}}
\newcommand{\clos}[1]{\overline{#1}}
\newcommand{\resid}[1]{\overline{#1}}
\newcommand{\cover}[1]{\widetilde{#1}}
\newcommand{\overbar}[1]{%
  \mkern 1.5mu\overline{\mkern-1.5mu#1\mkern-1.5mu}\mkern 1.5mu%
}

%  Defines a bordered matrix with square bracket delimiters instead of parentheses.
%  (http://tex.stackexchange.com/questions/55054)
%
\let\bbordermatrix\bordermatrix
\patchcmd{\bbordermatrix}{8.75}{4.75}{}{}
\patchcmd{\bbordermatrix}{\left(}{\left[}{}{}
\patchcmd{\bbordermatrix}{\right)}{\right]}{}{}

%  Calls one of the mathabx font families so that it is possible to use its symbols without making a global change.
%  (http://www.ctan.org/pkg/mathabx)
%  (http://tex.stackexchange.com/questions/14386)
%
\DeclareFontFamily{U}{mathb}{\hyphenchar\font45}
\DeclareFontShape{U}{mathb}{m}{n}{<5> <6> <7> <8> <9> <10> gen * mathb
<10.95> mathb10 <12> <14.4> <17.28> <20.74> <24.88> mathb12}{}
\DeclareSymbolFont{mathb}{U}{mathb}{m}{n}

% Defines circular arrows

\DeclareFontFamily{U}{mathb}{\hyphenchar\font45}
\DeclareFontShape{U}{mathb}{m}{n}{<5> <6> <7> <8> <9> <10> gen * mathb
<10.95> mathb10 <12> <14.4> <17.28> <20.74> <24.88> mathb12}{}
\DeclareSymbolFont{mathb}{U}{mathb}{m}{n}

%  Defines circular arrows.
\DeclareMathSymbol{\lcirclearrow}{0}{mathb}{'366}
\DeclareMathSymbol{\rcirclearrow}{0}{mathb}{'367}
\newcommand{\leftcirclearrow}{\mathrel{\ensuremath{\raisebox{0.1ex}{\scalebox{0.9}{\rotatebox[origin=c]{90}{$\lcirclearrow$}}}}}}
\newcommand{\rightcirclearrow}{\mathrel{\ensuremath{\raisebox{0.1ex}{\scalebox{0.9}{\rotatebox[origin=c]{270}{$\rcirclearrow$}}}}}}

% Semantic names for some common math Symbols.

%-----------------------------------%
%  Things Specific to Course Notes  %
%-----------------------------------%


%  Formatting for the table of contents. The first line allows for multi-column environments, the second line removes the heading "Contents".
\usepackage{multicol} \setlength{\columnsep}{3cm}
\makeatletter \renewcommand\tableofcontents{\@starttoc{toc}} \makeatother


%  Sets the page style.
\usepackage{fancyhdr}
\pagestyle{fancy}
\renewcommand{\headrulewidth}{0pt}
\renewcommand{\footrulewidth}{0.5pt}
\setlength{\headheight}{14pt}
\lfoot{\parbox[t]{1in}{\centering Last edited\\ \today}}
\cfoot{\parbox[t]{3in}{\centering \coursetitle}}
\rfoot{\parbox[t]{0.9in}{\centering Page \thepage\\ Lecture \arabic{lecture}}}


%  Sets the inputs for \maketitle.
\author{Lectures by \lecturer\\ Notes by \notetaker}
\title{\coursetitle}
\date{\institution\\ \courseterm}


%  Defines headings for each day's notes.
\newcommand{\classheader}[1]{\stepcounter{lecture}\newpage\section*{Lecture \arabic{lecture} (#1)}
\phantomsection \addcontentsline{toc}{section}{Lecture \arabic{lecture} (#1)}}
	
% Misc additions to Template

% http://tex.stackexchange.com/questions/18359
\pgfplotsset{compat=newest}

\newcommand{\Cinfty}{\ensuremath{C^{\infty}}}
\newcommand{\Crit}{\mathrm{Crit}}
\usepackage{mathtools}
\newcommand{\Or}{\mathrm{Or}}
\renewcommand{\Re}{\mathrm{Re}}
\renewcommand{\Im}{\mathrm{Im}}
\usepackage{mathrsfs}
\newtheorem*{examples}{Examples}
\newtheorem*{exercise}{Exercise}
\usepackage{pdfpages}
\newcommand{\Lie}{\mathrm{Lie}}
\newcommand{\Diffeo}{\mathrm{Diffeo}}

\newcommand{\connection}{\nabla}
\newcommand{\new}{\mathrm{new}}


\newcommand{\review}{{\huge\color{myred}{$\star$}}}


%---------------------------%
%  Hyperlinks and Metadata  %
%---------------------------%
%
% (this section must come last!)


%  hyperref enables for the creation of hyperlinks, and also specifies the metadata of the PDF file.
%  hyperxmp allows more metadata to be specified.
%  (http://www.ctan.org/pkg/hyperref)
%  (http://www.ctan.org/pkg/hyperxmp)
%  (http://tex.stackexchange.com/questions/41461)
%
\usepackage{hyperref}
\usepackage{hyperxmp}
\hypersetup{
pdfauthor={\notetaker},
pdftitle={\coursetitle},
pdfproducer={LaTeX},
%pdfcopyright={Copyright (C) \the\year\ \notetaker. This work is licensed under a Creative Commons Attribution-ShareAlike 3.0 Unported License. All attribution should be to \lecturer\ as the lecturer, and to \notetaker\ as the person taking these notes.},
pdfsubject={differential topology},
pdfkeywords={},
%pdflicenseurl={http://creativecommons.org/licenses/by-sa/3.0/},
colorlinks=true,
linkcolor=myred,
citecolor=mygreen,
urlcolor=myblue,
linktoc=page,
pdfstartview=FitH
}


%%%%%% SLOPE FIELDS

\pgfplotsset{ % Define a common style, so we don't repeat ourselves
    MaoYiyi/.style={
        width=0.6\textwidth, % Overall width of the plot
        axis equal image, % Unit vectors for both axes have the same length
        view={0}{90}, % We need to use "3D" plots, but we set the view so we look at them from straight up
        xmin=0, xmax=1.1, % Axis limits
        ymin=0, ymax=1.1,
        domain=0:xmax, y domain=0:xmax, % Domain over which to evaluate the functions
        xtick={0,0.5,1}, ytick={0,0.5,1}, % Tick marks
        samples=11, % How many arrows?
        cycle list={    % Plot styles
                gray,
                quiver={
                    u={1}, v={f(x)}, % End points of the arrows
                    scale arrows=0.075,
                    every arrow/.append style={
                        -latex % Arrow tip
                    },
                }\\
                red, samples=31, smooth, thick, no markers, domain=0:2\\ % The plot style for the function
        }
    }
}


%------------%
%  DOCUMENT  %
%------------%

\begin{document}
	
% TITLE

\maketitle
\thispdfpagelabel{Title}
\thispagestyle{empty}
\setcounter{page}{-1}
\vspace{0.3in}



%  Table of Contents
%
\begin{center}
\begin{minipage}[t]{0.9\textwidth}
\begin{multicols}{2}
\tableofcontents
\end{multicols}
\end{minipage}
\end{center}



\newpage
\thispdfpagelabel{-}
\thispagestyle{empty}

%% Introduction

\section*{Introduction}
Math 110.304 is one of the semi-important courses that is required/recommended for the engineering-based majors at Johns Hopkins University.

These notes are being live-TeXed, through I edot for Typos and add diagrams requiring the Ti\textit{k}Z package separately. I am using Texpad on Mac OS X.

I would like to thank Zev Chonoles from The University of Chicago and Max Wang from Harvard University for providing me with the inspiration to start live-TeXing my notes. They also provided me the starting template for this, which can be found on their personal websites.

Please email any corrections or suggestions to \expandafter\href{mailto:\notetakersemail}{\texttt{\notetakersemail}}.
	
	
\newpage

\classheader{2018-08-30}
\section*{Introduction}
\begin{example-N}
	Up to 6 units of two nutrients can be added to solution and we require the number of units of nutrient 2 has to be atleast (natural) logarithm of \# units of nutrient one. \underline{Goal:} choose $x_1$ = number units of nutrient 1, choose $x_2$ = number units of nutrient 2. To maximize expected height of plant $1 + x_1^2(x_2-1)^3 e^{-x_1-x_2}$
\begin{gather*}
	\text{Maximise } \quad 1 + x_1^2(x_2-1)^3 e^{-x_1-x_2}\\ 
	x_1 + x_2 \leq 6\\
	x_2 > \log x_1 \\
	x_1 \geq 0\\
	x_2 \geq 0
\end{gather*}
\begin{center}
	\begin{tikzpicture}
		\begin{axis}
		[xmin =0, xmax = 8,
		ymin = 0, ymax = 8,
		axis lines = middle, samples=100]	
	
		\addplot[mark=none, draw=black, thin, domain=0:6]{6 - x};
		\end{axis}
	\end{tikzpicture}
\end{center}
\begin{gather*}
	\text{generic optimization problem: say } \subseteq \mathbb{R}\\
		f: \underbrace{\mathbb{S}}_{\text{feasible region}} \rightarrow \underbrace{\mathbb{R}}_{\text{objective function}}
\end{gather*}
\begin{gather*}
	x^* \begin{bmatrix}
		2\\4
	\end{bmatrix} \quad \text{optimal}\\
	\text{however optimal objective function value is} 1.2677
\end{gather*}
\end{example-N}
\begin{definition-N}
	We say $x^*$ is an \underline{optimal solution} if
	\begin{itemize}
		\item $x^* \in \mathbb{S}$
		\item For any $y \in \mathbb{S}$
	\end{itemize}
\end{definition-N}
\begin{example-N}
	Find min $\log x$ s.t. $-\infty \leq x \leq 7 \rightarrow$ unbounded, has no solution.
\end{example-N}
\begin{example-N}
	Find min $\log x$ s.t. $1 < x \leq 7 \rightarrow$ bounded, but is also no solution. (as we can go 1.000001)
\end{example-N}
\begin{example-N}
	Find min $\log x$ s.t. $x > 1$, $x \leq 0.5 \rightarrow$ infeasible!!!
\end{example-N}
\begin{example-N}
	minimize $3 + (x-1)^2$ s.t. $1 \leq x \leq 3 \rightarrow$ feasible as optimal solution $x^* = 2$ is an interior part of feasible region.
	% TODO: Add Graph
	\begin{gather*}
		f(x) = 3 + (x-1)^2\\
		f'(x) = 0\\
		f''(x) > 0
	\end{gather*} 
\end{example-N}
\begin{example-N}
	min $3 + (x-2)^2$ s.t. $x \geq 10 \rightarrow$ optimal solution is $x^* = 10$. but $f'(x) \neq 0$. But is not an interior point - it is a boundary point of feasible region.\\
	% TODO: Draw this graph
\end{example-N}
\classheader{2018-08-31}
\begin{definition-N}
	$\forall x \in \mathbb{R}^n$ Euclidian length of $x$ is $||x|| = \bigg( \sum\limits_{i=1}^n x_i^2\bigg)^{\frac{1}{2}}$\\
	$\forall x,y \in \mathbb{R}^n$ Euclidian distance from $x$ to $y$ is $||x-y||$
\end{definition-N}
% TODO: draw vector diagram
\begin{definition-N}
	$\forall S \subseteq \mathbb{R}^n$ point $x \in S$ is an interior point of S if $J$ a neighborhood of $x$ which is a subset of $S$ point $S \in \mathbb{R}^n$ is a boundary point of S if every neighborhood of $x$ contains a point in $S$ and a point not in $S$
	% TODO: Draw illustration from Brice's notes
	\begin{itemize}
	\item Set $S \subseteq \mathbb{R}^n$ is \underline{open} if every point in $S$ is an interior point. \textsf{example: open ball.}
	\item  Set $S \subseteq \mathbb{R}^n$ is \underline{closed} if $S$ contains all boundary points of $S$.
\end{itemize}
 % TODO: Diagram from brice
 \emph{Note:} $\forall S \subseteq \mathbb{R}^n, S$ is open \textbf{iff} $S^c$ is closed.\\
In $(P)$ min $f(x)$ s.t. $x \in S$, suppose $x^* \in S$:
 \begin{itemize}
 	\item $x^*$ is a \underline{global maximizer} if $\forall y \in S$, $f(x^*) \leq f(y)$
 	\item $x^*$ is a \underline{strict} global minimizer if $\forall y \in S$, s.t. $y \neq x^*$, $f(x^*) < f(y)$
 	\item $x^*$ is a \underline{local minimizer} if $\exists$ any neighborhood $N$ of $x$ s.t. $\forall y \in N \cap S$, $f(x^*) \leq f(y)$
 	\item $x^*$ is a \underline{strict} local minimizer if $\exists$ any neighborhood $N$ of $x$ s.t. $\forall y \in (S \cap N)\underbrace{\backslash x^*}_{\text{besides}}$, $f(x^*) < f(y)$
 \end{itemize}
\end{definition-N}
\emph{Note:} If $S \subseteq \mathbb{R}^1$, $x^*$ is interior of $S$, $f$ is suitably differentiable at $x^*$.
\begin{center}
If $x^*$ is a \textsl{local minimizer} $\Rightarrow f'(x^*) = 0 \hspace{7em}$  $x^*$ is a \textsl{local maximizer} $\Rightarrow f'(x^*) = 0$	
\end{center}
\begin{center}
	{\Large \textbf{BUT} $f'(x^*) = 0 \quad \xcancel{\Longrightarrow} \quad x^*$ local min/max}
\end{center}

\begin{center}

	\begin{tikzpicture}
		\begin{axis}
		[xmin =-5, xmax = 5,
		ymin = -5, ymax = 5,
		axis lines = middle, samples=100, scale=0.7]	
	
		\addplot[mark=none, draw=black, thin, domain=-5:5]{x^3};
		\end{axis}
	\end{tikzpicture}
\end{center}
\textbf{BUT}
\begin{gather*}
	f'(x^*) = 0 \& f''(x^*) > 0 \Longrightarrow x^* \text{ strict local max}\\
	f'(x^*) = 0 \& f''(x^*) < 0 \Longrightarrow x^* \text{ strict local min}
\end{gather*}
If $S \subseteq \mathbb{R}^n$, $x^*$ interior part of $S$, $f$ strictly differentiable, $x^*$ local min/max $\Longrightarrow \nabla f(x^*) = \vec{0}$
\begin{equation*}
	\nabla f(x^*) = \vec{0} \quad {\large \&} \hspace{0.5em} \large[ ? \large] \Rightarrow x^* \text{ strict local min/max}
\end{equation*}

\classheader{2018-09-05}
\section*{Linear Programming}
\begin{example-N}
	\emph{Diet Problem}: You will pick levels of four ingredients for chicken feed.
	\begin{multicols}{2}
		\begin{itemize}
		\item $x_1$ = units of ingredient 1
		\item $x_2$ = units of ingredient 2
		\item $x_3$ = units of ingredient 3
		\item $x_4$ = units of ingredient 4 
	\end{itemize}
	\end{multicols}
	All are real numbers, so fractions allowed.
	Given minimum levels of 3 nutrients $\begin{bmatrix}
		6.2\\11.9\\10
	\end{bmatrix}$ = $\begin{bmatrix}
		\text{nutrient 1}\\
		\text{nutrient 2}\\
		\text{nutrient 3}\\
	\end{bmatrix}$\\
	Given how many units of nutrient per unit of ingredient\\
	\begin{tabular} {c||c|c|c|c}
	nutrients $\backslash$ ingredients & 1 & 2 & 3 & 4\\
	\hline 
		1 (protein) & 1.2 & 2.6 & 0 & 9.2\\
		2 (carbs) & 3.9 & 1 & .8 & 2\\
		3 (cholesterol) & 6 & 0 & 4 & 3.1
	\end{tabular}\\
	\textbf{Problem}: Find yand of ingredients that meet nutritional requirements cheaply as possible.\\
	Minimum Cost of the ingredients $\begin{bmatrix}
		6.2\\2\\1.6\\3.2
	\end{bmatrix}$
	\begin{gather*}
		\text{minimize} \qquad 6.2x_1 + 2x_2 + 1.6x_3 + 3.2x_4\\
		\begin{cases}
			1.2x_1 + 2.6x_2 + 0x_3 + 9.2x_4	\geq 6.2\\
		3.9x_1 + x_2 + 0.8x_3 + 2x_4 \geq 11.9\\
		3.9x_1 + 1x_2 + 0.8x_3 + 2 x_4 \geq 11.9\\
		6x_1 + 0x_2 + 4x_3 + 3.1x_4 \geq 11.9\\
		\end{cases}
		\text{and} \quad x_1, x_2, x_3, x_4 \geq 0
	\end{gather*}
\end{example-N}
What makes a ingredient list special?
\begin{enumerate}
	\item proportionality
	\item additivity
	\item divisibility
	\begin{itemize}
		\item fractions are allowed -- referring to $x_1, x_2, x_3$
	\end{itemize}
\end{enumerate}
\textbf{(Linear Programming)} in \underline{canonical form}, an example
\begin{gather*}
	\text{minimize} \qquad c_1x_1 + c_2x_2 + \cdots + c_nx_n\\
	\text{s.t.} a_{11}x_1 + a_{12}x_2 + \cdots + a_{1n}x_n \geq b_1\\
	a_{21}x_1 + a_{22}x_2 + \cdots + a_{2n}x_n \geq b_2\\
	\vdots\\
	a_{m1}x_1 + a_{m2}x_2 + \cdots + a_{mn}x_n \geq b_m\\
	\text{where} \qquad x_1, x_2, \cdots, x_n \geq 0\\
	\Updownarrow\\
	\text{min} \quad c^Tx \quad \text{s.t.} \quad A\vec{x} \geq \vec{b}, x \geq \vec{0}\\
	A \in \mathbb{R}^{mxn}, b \in \mathbb{R}^m, C \in \mathbb{R}^n, x \in \mathbb{R}^n
\end{gather*}
\begin{example-N}
	\emph{Transportation problem}: There are 3 electricity generation plants $\alpha, \beta, \gamma$  and two cities $u,v$. $\alpha, \beta, \gamma$ respectively produce $65.2, 98.6, 32.5$ units of electricity. $u,v$ respectively use $86.2, 110.1$ units of electricity\\ \textbf{Question:} How to supply transportation as cheaply as possible?
	\begin{center}
	\begin{tabular}{|c|c|c|}
	\text{generator to city} & \text{cost variable} & \text{cost}\\
	\hline
	$\alpha \rightarrow u$ & $x_1$ & 31.7\\
	$\alpha \rightarrow v$ & $x_2$ & 28.6\\
	$\beta \rightarrow u$ & $x_3$ & 17.6\\
	$\beta \rightarrow v$ & $x_4$ & 37.4\\
	$\gamma \rightarrow u$ & $x_5$ & 22.8\\
	$\gamma \rightarrow v$ & $x_6$ & 29.7\\
	\end{tabular}
	\end{center}
	\begin{gather*}
		\text{min} \qquad 31.7x_1 + 28.6x_2 + 17.6x_3 + 374x_4 + 22.8x_5 + 29.7x_6
		%TODO: Finish table
	\end{gather*}
	\begin{align*}
	\begin{rcases}
		x_1 + x_2 \leq 65.2 \qquad \alpha\\
		x_3 + x_4 \leq 98.6 \qquad \beta\\
		x_5 + x_6 \leq 32.5 \qquad \gamma
	\end{rcases} \text{electricity produced at } \alpha, \beta, \gamma\\
	\begin{rcases}
		u \qquad x_1 + x_3 + x_5  \geq 86.2\\
		v \qquad x_2 + x_4 + x_6 \geq 110.1
	\end{rcases} \text{Need for electricity for }u, v
	\end{align*} 
\end{example-N}
\textbf{(LP)} in standard form \qquad minimize $ c^Tx \qquad$ s.t. $Ax = b, \quad x \geq \vec{0}$\\
Here
\begin{gather*}
	A = \begin{bmatrix}
		1 & 1 & 0 & 0 & 0 & 0\\
		0 & 0 & 1 & 1 & 0 & 0\\
		0 & 0 & 0 & 0 & 1 & 1\\
		1 & 0 & 1 & 0 & 1 & 0\\
		0 & 1 & 0 & 1 & 0 & 1\\
	\end{bmatrix} \qquad b = 
	\begin{bmatrix}
		65.2\\
		98.6\\
		32.5\\
		86.2\\
		110.1
	\end{bmatrix}\\
	c^T = \begin{bmatrix}
		31.7 & 28.6 & 17.6 & 37.4 & 22.8 & 29.7
	\end{bmatrix}
\end{gather*}
Converting one form of (LP) to another\\
max $\rightarrow$ min $\qquad$ min $\rightarrow$ max $\quad \Longrightarrow \quad$ max $6x_1 + 3x_2 - 4x_3 \leftrightarrow \text{ min } -6x_1 - 3x_2 + 4x_3$
\begin{gather*}
	Ax \geq b \rightarrow \leq \\
	Ax \geq b \leftrightarrow \boxed{-A}x \leq \boxed{-b}
\end{gather*}
\textbf{Standard form $\rightarrow$ canonical form}
\begin{gather*}
\begin{rcases}
	\text{min} \hspace{0.5em}  c^Tx \hspace{0.5em} \\ \text{s.t.} \hspace{0.5em} Ax = b\\
	x \geq 0
\end{rcases}
\leftrightarrow 
\begin{rcases}
	\begin{cases}
	\text{min} \hspace{0.5em} c^Tx \hspace{0.5em} \\ \text{s.t.} \hspace{0.5em}Ax \geq b\\
Ax \leq b\\
x \geq \vec{0}
\end{cases}
\end{rcases} \leftrightarrow
\begin{rcases}
	\begin{cases}
	\text{min} \hspace{0.5em} c^Tx \\ \text{ s.t. } Ax \geq b\\
	-Ax \geq -b\\
	x \geq \vec{0}
\end{cases}
\end{rcases} \leftrightarrow
\begin{cases}
	\text{min } c^Tx \text{ s.t. }\\
	\begin{bmatrix}
		A\\
		-A
	\end{bmatrix} x \geq 
	\begin{bmatrix}
		b\\-b
	\end{bmatrix}\\
	x \geq \vec{0}
\end{cases}
\end{gather*}
\textbf{Canoconical form $\rightarrow$ Standard form}
\begin{gather*}
	\begin{split}
		a_{11}x_1 + a_{12}x_2 + a_{13} \geq b_1\\
		a_{21}x_1 + a_{22}x_2 + a_{23} \geq b_2\\
	\end{split} \quad \Longleftrightarrow \quad
	\begin{split}
		a_{11}x_1 + a_{12}x_2 + a_{13} + x_4 \qquad = b_1\\
		a_{21}x_1 + a_{22}x_2 + a_{23} \underbrace{ \qquad + x_5}_{\text{slack variables}}= b_2\\
	\end{split}
\end{gather*}

\classheader{2018-09-07}
\begin{enumerate}
	\setcounter{enumi}{4}
	\item \textbf{non-negativity $\rightarrow$ include in matrix}
	\begin{gather*}
		\begin{rcases}
		Ax \geq b\\
		x \geq \vec{0}	
		\end{rcases} \longrightarrow 
		\begin{bmatrix}
			A\\I
		\end{bmatrix} x \geq \begin{bmatrix}
			\vec{b}\\ \vec{0}
		\end{bmatrix}
	\end{gather*}
	Simplex algorithm only works with non-negative elements.
	\item \textbf{unconstrained sign $\rightarrow$ non-negativity}
	by \underline{substitution} 
	\begin{gather*}
		\underbrace{Z}_{\text{uncons}} := \underbrace{Z_1}_{\geq 0} - \underbrace{Z_2}_{\geq 0}\\
		\begin{rcases}
			\textit{min} \quad & 5x_1 + 6x_2\\
		\textit{s.t.} \quad & 2x_1 - 3x_2 \geq 9\\
		& x_1 + x_2 \geq -8\\
		& x_1 \geq 0 \qquad x_2 \text{\small unconst}
		\end{rcases} \rightarrow 
		\begin{cases}
			\begin{rcases}
			X_2 =\underbrace{X_2'}_{\geq 0}	 - \underbrace{X_2''}_{\geq 0}
			\end{rcases} \rightarrow
		\end{cases}
		\begin{cases}
			\textit{min} & 5x_1 + 6x_2' - 6x_2''\\
			\textit{s.t.} & 2x_1 - 3x_2' + 3x_2'' \geq 9\\
		& x_1 + x_2' - x_2'' \geq -8\\
		& x_1, x_2', x_2'' \geq 0
		\end{cases}
	\end{gather*}
	\item \textbf{Have constraints $Ax=b$}
	\begin{gather*}
		\text{illustration} \quad x_1, x_2, x_3 \begin{cases}
			2x_1 -2x_2 + 6x_3 = 8\\
			3x_1 + x_2 - x_3 =2
		\end{cases}
	\end{gather*}
	Unchanged by row opeartions
	\begin{itemize}
		\item swap rows
		\item multiply by non-zero constant
		\item add one row to another
	\end{itemize}
	\begin{gather*}
	\left[
		\begin{array}{ccc|c}
		1 & -1 & 3 & 4\\
		3 & 1 & -1 & 2	
		\end{array}
	\right] \xrightarrow{\text{rref}}
	\left[
		\begin{array}{ccc|c}
		1 & 0 & \frac{1}{2} & -\frac{3}{2}\\
		0 & 1 & -\frac{5}{2} & -\frac{5}{2}	
		\end{array}
	\right]
	\end{gather*}
	If row reduce $\left[ \begin{array}{c:c}
		A & b\\
	\end{array} \right] \rightarrow \left[ \begin{array}{c:c}
		A' & b'\\
	\end{array} \right]$ then $\exists$ invertible matrix $C$\\
	\begin{gather*}
		C\left[ \begin{array}{c:c}
		A & b\\
	\end{array} \right] \rightarrow \left[ \begin{array}{c:c}
		A' & b'
		\end{array} \right]
	\end{gather*}
	\textbf{Converse: } if $D$ is invertible, then $\left[ \begin{array}{c:c}
		A' & b'\\
	\end{array} \right]$ can be reduced to $D\left[ \begin{array}{c:c}
		A & b\\
	\end{array} \right]$
	\item Changing the order of columns in $Ax = b$, it doesn't matter as long as you change the order of the variables conformally
	\begin{gather*}
		\begin{rcases}
			\textit{min} & \begin{bmatrix}
				17\\83\\-11
			\end{bmatrix}^T\begin{bmatrix}
				x_1\\ x_2\\x_3
			\end{bmatrix}\\
			\textit{s.t.} & \begin{bmatrix}
				2 & -2 & 6\\
				3 & 1 & -1
			\end{bmatrix}
			\begin{bmatrix}
				x_1\\x_2\\x_3
			\end{bmatrix} = \begin{bmatrix}
				8\\2
			\end{bmatrix}
		\end{rcases} \longrightarrow
		\begin{cases}
			\textit{min} & \begin{bmatrix}
				83\\-11\\17
			\end{bmatrix}^T\begin{bmatrix}
				x_2\\x_3\\x_1
			\end{bmatrix}\\
			\textit{s.t.} & \begin{bmatrix}
				-2 & 6 & 3\\
				1 & -1 & 3
			\end{bmatrix}
			\begin{bmatrix}
				x_2\\x_3\\x_1
			\end{bmatrix} = \begin{bmatrix}
				8\\2
			\end{bmatrix}
		\end{cases}
	\end{gather*}
\end{enumerate}
\classheader{2018-09-12}
\subsection*{Half spaces, Hyperplanes, Polyhedral sets}
\textbf{Recall: } Suppose $p,x \in \mathbb{R}^n$. The inner product of $p,x$
\begin{gather*}
	p^Tx = \left||p|\right| \left||x|\right| \cos \theta\\
	\text{thus } p^Tx 
	\begin{cases}
		> 0 & \text{if $\theta$ acute}\\
		= 0 & \text{if $p \perp x$ (perpendicular)}\\
		< 0 & \text{if $\theta$ obtuse}\\
	\end{cases}
\end{gather*}
\begin{definition-N}
	Suppose $p \in \mathbb{R}^n$ non-zero. $\{\vec{x} \in \mathbb{R}^n : p^T x = 0 \}$ is a hyperplane through origin with normal vector $p$. $\{\vec{x} \in \mathbb{R}^n : p^T x \leq 0 \}$ is (associated) and (closed) half space.
\end{definition-N}
\textbf{Example}
	\begin{center}
		\begin{tikzpicture}
		\coordinate (A) at (0,-1);
  		\coordinate (B) at (0,4);
  		\coordinate (C) at (4,0);
  		\coordinate (D) at (-4,0);
		\coordinate (E) at (2,3);
		\coordinate (F) at (0,0);
		\coordinate (G) at (-3,2);
		\coordinate (H) at (3,-2);
  		%\draw [->] (A) -- (B) node [pos=.9, auto, swap] {$b_1$} node [pos=.9, auto] {$C$} ;
  		%\draw [->] (A) -- (C) node [pos=.9, auto] {$b_2$} ;
  		\draw[->, line width=0.5mm] (F) -- (E) node [pos = .9, auto, swap] {$\vec{p} = \begin{bmatrix}
  			2\\3
  		\end{bmatrix}$};
  		\draw [<->] (C) -- (D) node[pos = .9, auto, swap] {$x$};
  		\draw [<->] (A) -- (B) node[pos = .9, auto, swap] {$y$};
  		\draw [densely dotted] (G) -- (H) node[pos = .9, auto] {perp to p $\rightarrow$ hyperplane};
	\end{tikzpicture}
	\end{center}
Everything on either side of the hyperplane is the \emph{half spaces}.
\begin{definition-N}
	In general \emph{hyperplanes} in $\mathbb{R}^n$ are sets $\{x \in \mathbb{R}^n, p^Tx = \alpha \}$ for non-zero $p \in \mathbb{R}^n$, and $\alpha \in \mathbb{R}$
\end{definition-N}
\begin{definition-N}
	\emph{half spaces} in $\mathbb{R}^n$ are sets $\{x \in \mathbb{R}^n, p^Tx \geq \alpha \}$ for non-zero $p \in \mathbb{R}^n$, and $\alpha \in \mathbb{R}$
\end{definition-N}
\begin{example-N}
	\begin{align*}
		\text{(2D)} \hspace{4.8em} 2x_1 + 5x_2 = 6 \qquad & \text{hyperplane}\\
		2x_1 + 5x_2 \geq 6 \qquad & \text{half space}\\
		\text{(3D)} \qquad 3x_1 + 2x_2 - 7x_3 = 8 \qquad & \text{hyperplane}\\
		3x_1 + 2x_2 - 7x_3 \geq 8 \qquad & \text{half space}
	\end{align*}
\end{example-N}
\begin{definition-N}
	A \emph{polyhedron} (polyhedral space) is the intersection of finely may half spaces.
	\begin{example-N}
		$x \cdot Ax \geq b$ for $A \in \mathbb{R}^{mxn}, \quad b \in \mathbb{B}^m$
		\begin{gather*}
			\begin{matrix}
				\vec{p_1} \rightarrow\\
				\vec{p_2} \rightarrow\\
				\vdots\\
				\vec{p_m} \rightarrow
			\end{matrix}
			\begin{bmatrix}
				a_{11} & a_{12} & a_{13} & \ldots & a_{1n}\\
				a_{21} & a_{22} & a_{23} & \ldots & a_{2n}\\
				\vdots & \vdots & \ddots & \ddots & \vdots\\
				a_{m1} & a_{m2} & a_{m3} & \ldots & a_{mn}
			\end{bmatrix}
			\begin{bmatrix}
				x_1 \\ x_2\\ \vdots\\ x_n
			\end{bmatrix} \geq
			\begin{bmatrix}
				b_1\\ b_2\\ \vdots\\ b_m
			\end{bmatrix}
		\end{gather*}
		$\vec{x}$ satisfies $Ax \geq b$ \underline{iff} $\vec{x}$ satisfies ${p_1}^Tx \geq b_1$, and $\vec{x}$ satisfies ${p_2}^Tx \geq b_2$, $\ldots$, ${p_m}^Tx \geq b_m$
	\end{example-N}
\end{definition-N}
\begin{example-N}
	\begin{gather*}
		\begin{bmatrix}
			1 & -2\\
			-1 & -1\\
			1 & 0\\
			0 & 1
		\end{bmatrix}
		\begin{bmatrix}
			x_1\\ x_2
		\end{bmatrix} \geq
		\begin{bmatrix}
			-6 \\ -5 \\ 0\\0
		\end{bmatrix}
	\end{gather*}
	\begin{multicols}{2}
		\begin{enumerate}
			\item $x_1 - 2x_2 \geq -6$
			\item $-x_1 -x_2 \geq 5$
			\item $x_1 \geq 0$
			\item $x_2 \geq 0$
		\end{enumerate}	
		\end{multicols}
		\begin{center}
			\begin{tikzpicture}[scale=0.8]

    		\draw[gray!50, thin, step=0.5] (-6,-1) grid (6,6);
    		\draw[very thick,->] (-6,0) -- (6.2,0) node[right] {$x_1$};
		    \draw[very thick,->] (0,-1) -- (0,6.2) node[above] {$x_2$};

		    \foreach \x in {-6,...,6} \draw (\x,0.05) -- (\x,-0.05) node[below] {\tiny\x};
		    \foreach \y in {-1,...,6} \draw (-0.05,\y) -- (0.05,\y) node[right] {\tiny\y};

		    \fill[blue!50!cyan,opacity=0.3] (0,0) -- (0,3) -- (4/3,11/3) -- (5,0)--  cycle;

		    \draw (-6,0) -- node[above left,sloped] {\tiny$-x_1-2x_2\geq-6$} (6,6);
		    \draw (-1,6) -- node[above right,sloped] {\tiny$-x_1-x_2\geq-5$} (6,-1);
			\end{tikzpicture}
		\end{center}
		\textbf{Solving (LP) geometrically}\\
		\begin{gather*}
			\begin{split}
				\text{min} \qquad & \begin{bmatrix}
					1\\-3
				\end{bmatrix}^T \begin{bmatrix}
					x_1\\x_2
				\end{bmatrix}\\
				\text{s.t.} \qquad & \begin{bmatrix}
					1 & -2\\
					-1 & -1
				\end{bmatrix} \geq 
				\begin{bmatrix}
					-6\\-5
				\end{bmatrix}\\
				& x_1 \geq 0, x_2 \geq 0
			\end{split}\hspace{4em}
			\begin{split}
				\textbf{polyhedral sets:}\\
				x: Ax \leq b \equiv \quad & [(-A)x \geq (-b)]\\
				x: Ax = b \equiv \quad & \begin{bmatrix}
					Ax \geq b\\
					Ax \leq b
				\end{bmatrix} \quad \text{i.e.} \begin{bmatrix}
					A\\-A
				\end{bmatrix} x \geq \begin{bmatrix}
					b\\-b
				\end{bmatrix}\\
				x: \underbrace{Ax = b}_{x \geq 0} \equiv \quad & \begin{bmatrix}
					A\\-A\\I
				\end{bmatrix} x \geq \begin{bmatrix}
					b\\-b\\ \vec{0}
				\end{bmatrix}
			\end{split}
		\end{gather*}
		\begin{definition}
			\underline{$\alpha$-level set} is $\{x: f(x) = \alpha \}$	
		\end{definition}
		Here $\alpha$-level set is $x: x_1 - 3x_2 = \alpha$ \qquad rewrite: $x_2 = \underbrace{\frac{1}{3}}_{\text{slope}} x - \underbrace{\frac{\alpha}{3}}_{\text{y-int}}$	\\
		\begin{multicols}{2}
		\begin{center}	
		\begin{tikzpicture}[scale=0.5]
			\begin{axis}[
			xmin =-2, xmax=4,
			ymin=-2, ymax =4,
			axis lines=center,
			axis on top=true,
			domain=-2:4,
			xlabel={$x$},
    		ylabel={$y$},]
						
			\addplot[mark=none, draw=black, thin]{x/3 - 3/3} node [above left, pos=1] {$\alpha = 3$};
			\addplot[mark=none, draw=black, thin]{x/3 - 0/3} node [above left, pos=1] {$\alpha = 0$};
			\addplot[mark=none, draw=black, thin]{x/3 + 3/3} node [above left, pos=1] {$\alpha = -3$};
			\addplot[mark=none, draw=black, thin, red]{x/3 + 6/3} node [above left, pos=1] {$\alpha = -6$};
			\end{axis}
		
		\end{tikzpicture}
		\end{center}
			The solution is the level set with the least $\alpha$. In this case, $\alpha = -6$
		\end{multicols}
\end{example-N}
\classheader{2018-09-14}
\underline{Hyperplane}:- \quad $x: p^Tx = \alpha$ \hspace{5em} \underline{Halfspace}:- \quad $x: p^Tx \geq \alpha$\\
Polyhedral set - intersection of halfspaces. The example $Ax \geq b$
\begin{example-N}
	\begin{gather*}
		\begin{aligned}
			\text{min} && c^Tx\\
			\text{s.t.} && Ax = b\\
			&& x \geq 0
		\end{aligned} \hspace{8em}
		\begin{aligned}
			\text{min} && c^Tx\\
			\text{s.t.} && Ax \geq b\\
			&& x \geq 0
		\end{aligned}
	\end{gather*}
	Level set \quad $c^Tx = \alpha$
\end{example-N}
\begin{definition-N}
	Let's say $x,y \in \mathbb{R}^n$. A \underline{complex combination} of $xy$ is $\lambda x + (1-\lambda )y \in \mathbb{R}^n$ where $\lambda \in [0,1]$
\end{definition-N}
\begin{example-N}
	$x = \begin{bmatrix}
		1\\1
	\end{bmatrix}, \quad y = \begin{bmatrix}
		2\\3
	\end{bmatrix}, \quad \text{let } \lambda = \frac{1}{2}$\\
	\begin{tikzpicture}[scale=0.5]
			\begin{axis}[
			xmin =-1, xmax=2.5,
			ymin=-1, ymax =3.5,
			axis lines=center,
			axis on top=true,
			domain=-2:4,
			xlabel={$x$},
    		ylabel={$y$},]
						
			\draw[-] (1,1) -- (2,3);
			\fill[red] (1,1) circle (2pt);
			\fill[red] (1.5,2) circle (4pt) node[right, black] {$\quad \frac{1}{2}x + \frac{1}{2}y$};
			\fill[red] (2,3) circle (2pt);
			\end{axis}
		\end{tikzpicture}
\end{example-N}
\begin{note}
In general, a convex combination of $x,y$ are points on the line segment from $x$ to $y$ as $\lambda x + (1-\lambda)y = y + \lambda(x - y)$
\end{note}
\begin{definition-N}
	$\mathcal{S} \subseteq \mathbb{R}^n$ is \underline{convex} if $\forall x,y \in \mathcal{S}, \forall x \in [0,1] \quad \lambda x + (1-\lambda)y \in \mathcal{S}$
	\begin{example-N}
		Every polyhedron $\mathcal{S} := \{x \in \rn: Ax \geq b\}$ is convex since if $y,z \in \sss$ [i.e. $Ay \geq b$, $Ax \geq b$] and $\lambda \in [0,1]$, then
		\begin{gather*}
			A(\lambda y + (1-\lambda)z) = \lambda A y + (1-\lambda) Az \geq \lambda b + (1 - \lambda)b = b\\
			\lambda y + (1-\lambda)z \in \sss
		\end{gather*}
	\end{example-N}
\end{definition-N}
\begin{definition-N}
	Let $\sss$ be a convex set. $X$ is an \underline{extreme point of $\sss$} if $X = \lambda y + (1-\lambda)z$ for, $y,z \in \sss$, $\lambda \in (0,1)$ $\Rightarrow x = y = z$
	\begin{example-N}
		Consider
		\begin{gather*}
			\begin{aligned}
				\text{min} && 10x_1 + 10x_2 + 0x_3 - 3x_4 - 5x_5 - 3x_6\\
				\text{s.t.} && -x_1 + 2x_2 + 3x_3 +6x_4 +9x_5 +8x_6 = 26\\
				&& -2x_1 + 3x_2 + x_3 +x_4 +6x_5 +8x_6 = 17\\
				&& x_1 + x_2 -x_3 +x_4 +x_5 +3x_6 = 1\\
				&& x_1, x_2, x_3, \ldots, x_6 \geq 0
			\end{aligned}
		\end{gather*}
		What are feasible $x$? What if \underline{\textbf{JUST FOR NOW}} we set $x_4, x_5, x_6 = 0$
		\begin{gather*}
			\underbrace{\begin{bmatrix}
				-1 & 2 & 3\\
				-2 & 3 & 1\\
				1 & 1 & -1
			\end{bmatrix}}_{B \rightarrow \text{"Basic"}}
			\underbrace{\begin{bmatrix}
				x_1\\x_2\\x_3
			\end{bmatrix}}_{x_B \rightarrow \text{basic variable}} = 
			\underbrace{\begin{bmatrix}
				26\\17\\1
			\end{bmatrix}}_{b}
		\end{gather*}
		Lucky \# 1 $\rightarrow B$ is invertible
		\begin{gather*}
			B^{-1} = \frac{1}{13} \begin{bmatrix}
				4 & -5 & -1\\
				1 & 2 & 5\\
				5 & -3 & -1
			\end{bmatrix}\\
			\text{so} \quad x_B = B^{-1}b = \begin{bmatrix}
				2\\5\\6
			\end{bmatrix}\\
			\text{Hence the basic feasible solutions are} \begin{bmatrix}
				x_1\\x_2\\x_3\\x_4\\x_5\\x_6
			\end{bmatrix} = \begin{bmatrix}
				2\\5\\6\\0\\0\\0
			\end{bmatrix}
		\end{gather*}
	\end{example-N}
\end{definition-N}
\classheader{2018-09-17}
Continuous from last lecture. If we choose $x_4 = \frac{1}{4}$, $x_5 = \frac{1}{10}$, $x_6 = \frac{1}{5}$
\begin{gather*}
	\begin{bmatrix}
		-1 & 2 & 3\\
		-2 & 3 & 1\\
		1 & 1 & -1
	\end{bmatrix}
	\begin{bmatrix}
		x_1\\
		x_2\\
		x_3
	\end{bmatrix} = 
	\underbrace{\underbrace{\begin{bmatrix}
		26\\
		17\\
		1
	\end{bmatrix}}_{\text{b}} - 
	\underbrace{\begin{bmatrix}
		6 & 9 & 8\\
		1 & 6 & 8\\
		1 & 1 & 3\\
	\end{bmatrix}}_{\text{N - nonbasic}}
	\begin{bmatrix}
		\frac{1}{10}\\
		\frac{1}{10}\\
		\frac{1}{5}
	\end{bmatrix}}_{\begin{bmatrix}
		229/10\\
		143/10\\
		1/5\\
	\end{bmatrix}}
\end{gather*}








	
\end{document}
