\classheader{2018-08-30}
\section*{Survey Sampling}
We have a \underline{population of objects} under study (people, animals, places, etc.). We will consider a single numerical measurement associated to object $i: x_i$
\begin{example}
	$N = 5000, x_i=$ height of person $i$, Population size = N. We denote population measurements $\{x_1, x_2, \cdots, x_N\}$\\
	Compute population quantities:
	\begin{multicols}{2}
		\begin{itemize}
		\item population total $\tau = \sum\limits_{i=1}^N x_i$
		\item population mean $\mu = \frac{\tau}{N}= \frac{\sum\limits_{i=1}^N x_i}{N}$
	\end{itemize}
	\end{multicols}

\end{example}
\textbf{Note: } $\tau$ and $\mu$ are \underline{population parameters}, their computation depends on all the population data.
\begin{question}
	 How to estimate $\tau$ and $\mu$ based on a sample of observation from this population?
\end{question}
\underline{Classical Answer:} Choose a "random" sample of objects and associated measurements denoted $\{x_1, x_2, \cdots, x_n \}$. \emph{Note: } capital $X_i$ denote random variables.\\
Whiter "Random"? Two types of ways to sample:
\begin{multicols}{2}
	\begin{itemize}[label={--}]
	\item without replacement
	\item with replacement
\end{itemize}
\end{multicols}
\begin{claim-N}
	If $X_i$ are drawn without replacement, then the distribution of $X_1$ and $X_2$ are identical. Is this true? \textbf{In fact, \underline{\Large it is}} $\Rightarrow$ They are \textbf{\underline{\large NOT}} independent but they are identically distributed.
\end{claim-N}
\begin{center}
	P(Ace in Pos 1) = P(Ace in Pos 2) = $\frac{4}{52}$
\end{center}
\subsubsection*{Combinatorial Approach}
"well-shuffled deck" $\leftrightarrow$ all $52!$ rearrangements of the card are equally likely. How many rearrangements have ace at pos 1? \underline{$4 \cdot 51!$}
\begin{equation*}
	P(A_1) = \frac{4 \cdot 51!}{52!} = \frac{4}{52} = P(A_2) = P(A_{19}) = P(A_{36})
\end{equation*}
\begin{question}
	If $X_1$ and $X_2$ are identically distributed, then how do they differ between corresponding draws with replacement?
\end{question} 
\begin{answer}
	Independence. We can have Random Variables that are identically distributed and not independent. Note if independent, $P(A_2 |A_1) = P(A_2)$.
\end{answer}
\begin{equation*}
	\begin{split}
		\textbf{with replacement}\\
		P(A_1) = \frac{4}{52}, \quad
		P(A_2) = \frac{4}{52}\\
		P(A_2 | A_1) = \frac{4}{52}
	\end{split} \hspace{10em}
	\begin{split}
		\textbf{without replacement}\\
		P(A_1) = \frac{4}{52}, \quad
		P(A_2) = \frac{4}{52}\\
		P(A_2 | A_1) = \frac{3}{51}
	\end{split}
\end{equation*}
We can see from this that depending on sampling method, we gain or lose independence. In the finite population sampling method, we have $1, \ldots, N$ objects we care about. \\
\textbf{Loss of Independence} when choosing sampling method is important. 